%%%%%%%%%%%%%%%%%%%%%%%%%%%%%%%%%%%%%%%%%%%%%%%%%%%%%%%%%%%%%%%%%%%%
%% I, the copyright holder of this work, release this work into the
%% public domain. This applies worldwide. In some countries this may
%% not be legally possible; if so: I grant anyone the right to use
%% this work for any purpose, without any conditions, unless such
%% conditions are required by law.
%%%%%%%%%%%%%%%%%%%%%%%%%%%%%%%%%%%%%%%%%%%%%%%%%%%%%%%%%%%%%%%%%%%%

\documentclass[
  digital, %% This option enables the default options for the
           %% digital version of a document. Replace with `printed`
           %% to enable the default options for the printed version
           %% of a document.
  notable,   %% Causes the coloring of tables. Replace with `notable`
           %% to restore plain tables.
  nolof,     %% Prints the List of Figures. Replace with `nolof` to
           %% hide the List of Figures.
  nolot,     %% Prints the List of Tables. Replace with `nolot` to
           %% hide the List of Tables.
  nocover
  %% More options are listed in the user guide at
  %% <http://mirrors.ctan.org/macros/latex/contrib/fithesis/guide/mu/fi.pdf>.
]{fithesis3}
%% The following section sets up the locales used in the thesis.
\usepackage[resetfonts]{cmap} %% We need to load the T2A font encoding
\usepackage[T1,T2A]{fontenc}  %% to use the Cyrillic fonts with Russian texts.
\usepackage[
  main=english, %% By using `czech` or `slovak` as the main locale
                %% instead of `english`, you can typeset the thesis
                %% in either Czech or Slovak, respectively.
  english,czech %% The additional keys allow
]{babel}        %% foreign texts to be typeset as follows:
%%
%%   \begin{otherlanguage}{german}  ... \end{otherlanguage}
%%   \begin{otherlanguage}{russian} ... \end{otherlanguage}
%%   \begin{otherlanguage}{czech}   ... \end{otherlanguage}
%%   \begin{otherlanguage}{slovak}  ... \end{otherlanguage}
%%
%% For non-Latin scripts, it may be necessary to load additional
%% fonts:
\usepackage{paratype}
\def\textrussian#1{{\usefont{T2A}{PTSerif-TLF}{m}{rm}#1}}
\newcommand{\symbiotic}{\textsc{Symbiotic}\xspace}
%%
%% The following section sets up the metadata of the thesis.
\thesissetup{
    date          = \the\year/\the\month/\the\day,
    university    = mu,
    faculty       = fi,
    type          = mgr,
    author        = Martina Vitovská,
    gender        = f,
    advisor       = {doc. RNDr. Jan Strejček, Ph.D.},
    title         = {Instrumentation of LLVM IR},
    TeXtitle      = {Instrumentation of LLVM IR},
    keywords      = {instrumentation, static analysis, program analysis, LLVM, memory safety},
    TeXkeywords   = {instrumentation, static analysis, program analysis, LLVM, memory safety},
    thanks        = {Most of all, I would like to thank my advisor Jan Strejček and my consultant Marek Chalupa for their guidance, patience and the time they spent discussing the thesis with me. I would also like to thank my family and friends for their support when I was working on this thesis.},
    abstract      = {The thesis is focused on instrumentation of LLVM IR. First, we give an insight into existing tools that instrument a code to perform an analysis on it. Then we implement and describe a new tool for instrumentation of LLVM IR that is configurable by a~user and can use information from various static analyses. We create a configuration for checking memory safety and reduce the number of inserted checks by employing a pointer analysis and dividing it into phases. At the end, we evaluate the instrumentation for checking memory safety integrated in a verification tool called \symbiotic on a set of benchmarks. },
}

\usepackage{caption}
\newcommand{\thenalg}{\thechapter .\arabic{nalg}}
\DeclareCaptionLabelFormat{algocaption}{Algorithm \thenalg} % defines a new caption label as Algorithm x.y

\usepackage[backend=biber]{biblatex}
\addbibresource{references.bib}

\usepackage{makeidx}      %% The `makeidx` package contains
\makeindex                %% helper commands for index typesetting.
%% These additional packages are used within the document:
\usepackage{paralist} %% Compact list environments
\usepackage{amsmath}  %% Mathematics
\usepackage{amsthm}
\usepackage[table]{xcolor}
\usepackage{amsfonts}
\usepackage{url}      %% Hyperlinks
\usepackage{listings} %% Source code highlighting
\lstset{
  basicstyle      = \ttfamily,%
  identifierstyle = \color{black},%
  keywordstyle    = \color{blue},%
  keywordstyle    = {[2]\color{cyan}},%
  keywordstyle    = {[3]\color{olive}},%
  stringstyle     = \color{teal},%
  commentstyle    = \itshape\color{magenta}}
\usepackage{floatrow} %% Putting captions above tables
\floatsetup[table]{capposition=top}

\usepackage{tikz}
\usetikzlibrary{shapes,arrows}
\usepackage{smartdiagram}
\usesmartdiagramlibrary{additions}
\usepackage{styles/llvm}
\usepackage{styles/nasm}

\usepackage{xspace}
\usepackage{xcolor}
\usepackage{subcaption}

\newcommand{\todo}[1]{\textcolor{red}{#1}}
\newcommand{\llvm}{\textsc{llvm}\xspace}
\newcommand{\klee}{\textsc{Klee}\xspace}
\newcommand{\stacklist}{\texttt{stack\_list}\xspace}
\newcommand{\globalslist}{\texttt{globals\_list}\xspace}
\newcommand{\heaplist}{\texttt{heap\_list}\xspace}
\newcommand{\dealloclist}{\texttt{deallocated\_list}\xspace}
\newcommand{\llvmpin}{\textsc{LLVMPin}\xspace}
\newcommand{\clang}{\textsc{Clang}\xspace}

\lstnewenvironment{algorithm}[1][] %defines the algorithm listing environment
{
    \refstepcounter{nalg} %increments algorithm number
    \captionsetup{labelformat=algocaption,labelsep=colon} %defines the caption setup for: it ises label format as the declared caption label above and makes label and caption text to be separated by a ':'
    \lstset{ %this is the stype
        mathescape=true,
        frame=lines,
        numbers=left,
        numberstyle=\tiny,
		numbersep=5pt,
        basicstyle=\footnotesize,
        keywordstyle=\color{black}\bfseries,
        keywords={,input, output, return, datatype, in, if, else, foreach, while, begin, end, then,and, or, }
        numbers=left,
        xleftmargin=1em,
		framexleftmargin=1em,
        #1 % this is to add specific settings to an usage of this environment (for instnce, the caption and referable label)
    }
}
{}

\begin{document}
\chapter{Introduction}
As our lives became highly dependent on computers, the question of software
reliability turned to be a serious issue. If we take into consideration the
size and the complexity of today's computer programs, it is almost impossible
to find all the bugs in a code manually. The need for an automatic program
analysis is therefore increasing.

We can distiguish two types of program analysis: dynamic and static. Whereas
dynamic analysis is performed by executing a program, static analysis
inspects a representation of a program (e.g.~a~source code) and analyzes it
without actually running it.

One of the techniques used in the context of both dynamic and static program
analysis is instrumentation. Instrumenting a program means inserting a new code
into an existing code in order to gather information relevant for a program
analysis. The new code usually does not change the behaviour of the program.
Instrumentation is used for example in profilers which insert instructions for
gathering information about time or memory consumption. It is also used by many
tools that check memory safety of programs as it is possible to track the state
of the memory with the inserted code.

The aim of this thesis is to present an overview of existing tools for
instrumentation of LLVM IR bitcode and to design a new tool for
instrumentation of LLVM IR. The tool should be configurable and should offer the
possibility to use results of external static analyses to reduce the amount of
newly inserted code. Another goal is to create a~configuration for finding
memory safety errors such as double free, invalid dereference, or
use-after-free-error.

The new tool is implemented as an independent module, which means it can be
combined with any tool that performs the actual analysis on the instrumented
program. We employed the instrumentation in \symbiotic~\cite{Symbiotic}, which
is an open-source tool for static analysis of sequential programs in LLVM.
\symbiotic is based on three techniques: instrumentation, program slicing, and
symbolic execution. Instrumentation is used to reduce the problem of checking
some property violations (e.g. checking for NULL pointer dereference) to
a~reachability problem. In other words, a given program is instrumented such
that the original program violates the considered property if and only if some
error location is reachable in the instrumented program. Program
slicing~\cite{weiser} is a technique that removes the code that is not relevant
for the reachability of error locations. Therefore, a subsequent analysis is
performed on a smaller amount of code and is thus faster. In \symbiotic,
the sliced program is passed to a symbolic executor~\cite{King} that goes
through every possible execution path of a program and looks whether some
error location is reachable.

\section{Structure of the thesis}

The structure of this thesis is following: In Chapter~\ref{chap:llvm} we give
an insight into LLVM and introduce basic instructions of LLVM IR.
Chapter~\ref{chap:tools} lists the existing tools for instrumenting LLVM IR and
tools that use instrumentation of LLVM IR for analysis of programs. We discuss
benefits and drawbacks of these tools. In Chapter~\ref{chap:instr} the general
approach of our configurable instrumentation is described together with the
configuration files that must be provided. Chapter~\ref{chap:memsafety} deals
with a~configuration of instrumentation for checking memory safety errors.
Besides the basic approach, we also propose an enhancement with a~pointer
analysis and with staged instrumentation to make the whole verification process
faster. Chapter~\ref{chap:eval} presents experimental evaluation of the
instrumentation designed for memory safety implemented in \symbiotic. We
compare the original and the enhanced versions of instrumentation on a set of
benchmarks from SV-COMP
2018~\footnote{\url{https://sv-comp.sosy-lab.org/2018/benchmarks.php}}.  In
Chapter~\ref{chap:future} we suggest future development of the implemented tool
and configurations. Chapter~\ref{chap:conclusion} is a conclusion of this
thesis.


\chapter{LLVM}\label{chap:llvm}
LLVM~\cite{llvm} is an open source project that provides compiler technologies
designed to be independent of a target architecture. The project has several
parts: LLVM intermediate representation (LLVM IR, often shorten to LLVM) for
representation of a source code, LLVM Core libraries for code generation,
optimization and transformation, and tools such as Clang compiler, LLVM linker
\texttt{llvm-link} or LLVM optimizer \texttt{opt}. Another important part of
LLVM project is LLVM Pass Framework, since LLVM passes are programs used for
transformations and optimizations of LLVM code. In this Chapter, we focus on
LLVM IR which can be used in three different forms: human readable
representation, bitcode representation and an in-memory compiler IR. For the
sake of simplicity, we will work only with the human readable form as all the
forms are equal and we will refer to it as LLVM from this point on.

\begin{figure}[h]
 \lstinputlisting[language=llvm,style=nasm]{examples/llvm.ll}
 \caption{An example of an LLVM module with function \texttt{main} that calls
 function \texttt{foo} which allocates an array of ten integers and stores the
 value of the global variable \texttt{number} in the first field of the array.}
 \label{fig:llvm_example}
\end{figure}

\section{LLVM structure} %TODO rename the chapter to LLVM IR?

The high-level structure of an LLVM program consists of modules. Each module
corresponds to one compilation unit. More modules can be linked together into
one module with the LLVM linker. They contain global variables and functions.

Each function begins with a \texttt{define} keyword and is composed of basic
blocks. Basic blocks are sequences of instructions, that have a single entry node
and a single exit node and there is no branching in them. Basic blocks form
a control flow graph for a function. In Figure~\ref{fig:llvm_example} we give
an example of an LLVM module with two function definitions, each containing one
basic block.

LLVM is in static single assignment form~\cite{ssa} (SSA form), which means that each
variable can be assigned only at one location in the code. It uses:

\begin{itemize}
    \item global identifiers that begin with the '@' character for functions
    and global variables, for example global variable \texttt{@number} in
    Figure~\ref{fig:llvm_example}, and
    \item local identifiers that begin with the '\%' character for register
    names and types, for example a local variable \texttt{\%a} in
    Figure~\ref{fig:llvm_example}.
\end{itemize}

As values can be named or unnnamed, identifiers for unnamed values are
represented as unsigned numbers with a corresponding prefix (e.g. local
identifier \texttt{\%1} in Figure~\ref{fig:llvm_example}).

\section{LLVM types}

LLVM is a strongly typed language. There are simple types such as integer type
(e.g. \texttt{i32} for a 32-bit integer), floating point types (e.g.
\texttt{float} or \texttt{double}) and pointer type (e.g. \texttt{i32*} for a
pointer to a 32-bit integer), and also types for vectors (e.g. \texttt{<10 x
i32>} for a vector of ten 32-bit integers), arrays (e.g. \texttt{[10 x i32]}
for an array of ten 32-bit integers) and structures (e.g. \texttt{\{i32, i32\}}
for a pair of 32-bit integers). Observe that in Figure~\ref{fig:llvm_example}
the global variable \texttt{number} is a 32-bit integer and
\texttt{a} is an array of ten 32-bit integers.

\section{Basic instructions}

In this section, we describe the most important instructions for this work. The
complete list of instructions can be found in the LLVM
documentation\footnote{\url{https://llvm.org/docs/LangRef.html}}.

A terminator instruction is used as the last instruction of each basic block
and it determines how the control flow will continue. For our purpose we will
describe only two of the terminator instructions: \texttt{ret} and \texttt{br}.

\begin{description}
\item[\texttt{ret}] is used to return from a function to a basic block
from which the function was called. It has one optional argument that represent
a return value of the function. In Figure~\ref{fig:llvm_example} we can see the
two variants of this instruction: in function \texttt{foo}, there is a
\texttt{ret void} instruction, because this function does not return any value,
whereas function \texttt{main} returns 0 (\texttt{ret i32 0}).

\item[\texttt{br}] determines which basic block from the current function
will follow. It represents either conditional branching:

\texttt{br i1 <condition>, label <true>, label <false>},

transfering the control flow to \texttt{true} block if the \texttt{condition} holds and to
\texttt{false} block otherwise, or uncoditional branching:

\texttt{br label <b>},

transfering the control flow to the block \texttt{b} unconditionally.
\end{description}

The most important instructions for working with memory are \texttt{load}
\texttt{store}, \texttt{alloca} and \texttt{getelementptr}:
\begin{description}
\item[\texttt{load}] is used to read from memory specified by its operand,
\item[\texttt{store}] is used to write to memory and has two operands: a value
to store and address of a target memory.
\item[\texttt{alloca}] allocates memory on the stack.
\item[\texttt{getelementptr}] gets the address of some element of an aggregate
\end{description}
We can see the usage of these instructions in Figure~\ref{fig:llvm_example} in
function \texttt{foo}, where an array \texttt{a} of ten 32-bit integers is
allocated on the stack with the \texttt{alloca} instruction on line 4, value of
global variable \texttt{number} is loaded and named as \texttt{\%1} on line 5
and later stored to \texttt{\%2} on line 8, which is the address of the fifth
element of \texttt{a} since the \texttt{getelementptr} instruction was used on
line 6.

Other relevant instructions are \texttt{call} instruction and \texttt{bitcast}
instruction:
\begin{description}
\item[\texttt{call}] instruction calls a function given as its
operand together with function's arguments,
\item[\texttt{bitcast}] converts a given value to a given type, e.g.~\texttt{bitcast i8 1 to
i32} converts 8-bit integer of value 1 to 32-bit integer of value 1.
\end{description}
 We can find an example of a \texttt{call} instruction in
 Figure~\ref{fig:llvm_example} in the function \texttt{main} which calls
 function \texttt{foo} on line~15. This instruction can also be used for
 dynamic memory allocation by calling a function for dynamic allocation, e.g.
 \texttt{malloc}.

Binary operator instructions have two operands of the same type and return a
result of an operation on these operands, for example \texttt{add} instruction
for addition or \texttt{sub} instruction for subtraction. There are usually two
versions of these instructions: one for integer values and one for floating
point values.

Bitwise binary instructions are used for bitwise operations. This category
contains for example bitwise logical operators like \texttt{and} instruction,
\texttt{or} instruction, etc.




\chapter{Overview of Existing Tools}\label{chap:tools}
\section{LLVMPin Instrumentation Framework}

\url{http://eces.colorado.edu/~blomsted/llvmpin/llvmpin.html}
\medskip

LLVMPin is a framework that simplifies implementing tools for LLVM~IR
instrumentation. The framework basically consists of two scripts:

\begin{itemize}
    \item \texttt{llvmpin-gen} runs the LLVMPin tool implemented by
          user and instruments a given code,
    \item \texttt{llvmpin} generates a new
	  bitcode after instrumentation and from the bitcode it produces an
          executable.
\end{itemize}

The user has to implement a LLVMPin tool, which is a C++ code that must contain three
parts:

\begin{itemize}
    \item \texttt{REQUIRE} routine with a list of existing LLVM
          passes\footnote{\url{http://llvm.org/docs/WritingAnLLVMPass.html}} required by
          the LLVMPin tool,
    \item \texttt{INSTRUMENT} routine that inserts calls to analysis functions
          defined by user,
    \item analysis part that contains definitions of analysis functions and a
          routine for analysis setup.
\end{itemize}

The framework provides the user with API that adds several functions and macros
that are useful for the instrumentation, for example, one of the basic
functions provided by the API is a function \texttt{INS\_InsertCall} that adds
a call to an analysis function. However, although the LLVMPin framework is one
of the instrumentation tools that are general and configurable, it does not
simplify the work for the user in terms of writing a code that performs the
core of the instrumentation process.


\section{LLVM-IR-MemProtect}

\url{https://github.com/toaarnio/llvm-ir-memprotect}

\section{LLVM IR Trace Profiler}

\url{https://github.com/ysshao/LLVM-Tracer#llvm-ir-trace-profiler-llvm-triacer-12}

\section{LLVM Instrumentation Pass}

\url{https://github.com/imdea-software/LLVM\_Instrumentation\_Pass}

\section{Other Tools}

In this section, we describe tools that are not designed for general
instrumentation but use instrumentation over LLVM for various purposes.

\url{https://clang.llvm.org/docs/UsersManual.html#id34}

Clang compiler provides several sanitizers that use instrumentation to insert
runtime checks for various errors and undefined behaviour. It offers a memory
error detection with AddressSanitizer
\textcolor{red}{\url{https://clang.llvm.org/docs/AddressSanitizer.html}}
(out-of-bounds dereferences, use-after-free, double-free, etc.), a detection of
data races in parallel programs with ThreadSanitizer or undefined behaviour
detection with UndefinedBehaviorSanitizer (e.g. integer overflows). All these
sanitizers modify the code during compilation and insert run-time checks.
Disadvantage is slowdown and memory overhead these tool introduce.

SAFECode (Static Analysis For safe Execution of Code)
\url{http://safecode.cs.illinois.edu/index.html} is a project developed at
University of Illinois that checks memory safety properties, such as accessing
valid memory locations, out-of-bounds accesses, invalid frees and detection of
dangling pointers. Like Clang sanitizers it inserts run-time checks at
compile-time. Moreover, it uses static analysis to minimize the number of
inserted run-time checks.

Another tool that uses instrumentation is Map2Check. Like Symbiotic, it uses
KLEE as symbolic executor and instruments the code that is being analyzed to
check memory safety properties.


\chapter{Configurable Instrumentation}\label{chap:instr}
The basic idea of our instrumentation tool is simple. Since it is required to
be configurable, it needs to be supplied with two files: a JSON file with
instrumentation rules and a file with definitons of instrumentation functions.
The tool then loads a whole module from a given LLVM code and begins to process
phases. In each phase it goes through all functions and it first applies rules for
inserting instructions at their entry points and above their terminator
instructions. Then it goes through instructions of the current function and it
looks if they match with the sequences described in instrumentation rules. If a
match is found, the current sequence of instructions is instrumented above or
under according to the instrumentation rules. At the very end, a rule for
global variables is applied.

\section{Plugins}

The instrumentation module can be extended by plugins that can reply to queries
derived from conditions belonging to rules, which enables us to instrument
conditionally. A condition is a conjunction of predicates over values from the
program. Since static analyses provided by plugins are supposed to be
over-approximating, the predicates are set to reason about possibilities (e.g.
may the pointer be invalid?). Therefore answering true is a conservative answer
if an analysis does not have enough information to refute the predicate. A rule
is applied if all plugins answer that the given condition is satisfied. If at
least one plugin says that the condition is not satisfied, the rule is not
applied at the matched sequence and no instrumentation is performed.

To define a list of possible conditions, all plugins have to implement the same
interface.
\textcolor{red}{describe interface and how it works}

\section{Configuration}

\lstset{
    basicstyle=\footnotesize,
    string=[s]{"}{"},
    stringstyle=\color{blue},
    comment=[l]{:},
    commentstyle=\color{black},
}

\begin{figure}[h]
\lstinputlisting{examples/json_example2.json}
\caption{Example of an \texttt{globalVariablesRule} in a JSON configuration file.}
\label{fig:json_example2}
\end{figure}

\begin{figure}[h]
\lstinputlisting{examples/json_example.json}
\caption{Example of an \texttt{instructionRule} in a JSON configuration file.}
\label{fig:json_example}
\end{figure}

The JSON file contains following fields:

\medskip
\begin{itemize}
\item \texttt{file:} Path to a a file with definitions of instrumentation functions.
\item \texttt{analyses:} List of paths to analyses plugins.
\item \texttt{flags:} List of flags that can be set during instrumentation.
\item \texttt{globalVariablesRule:} Rules for instrumenting global
  variables. \textcolor{red}{TODO}
\item \texttt{phases:} List of instrumentation phases. Each phase contains a
  list of \texttt{instructionRules}. Each rule is described with several fields:
  \begin{itemize}
    \item \texttt{findInstructions}: Sequence of instructions we are searching
    for. For each instruction in the sequence, we need to fill in an
        \texttt{instruction} field, that specifies a name of the instruction,
        \texttt{returnValue} that enables to remember the return value of the
        instruction in a given variable (can be set to "*" if the return value
        is not needed), and \texttt{operands} that enables either to match the
        operands or to remember the operand values in given variables. We can
        also optionally fill in fields \texttt{getSizeTo} and
        \texttt{getPointerInfoTo}. \texttt{getSizeTo} can be used only with
        \texttt{load}, \texttt{store} or \texttt{alloca} instruction and it
        stores size of the type of the value that is being loaded, stored or
        allocated to the given variable. \texttt{getPointerInfoTo} can be used
        only with \texttt{load} or \texttt{store} and it stores two values to
        given variables (if possible): size of the allocated memory to which
        the dereferenced pointer points to and corresponding \texttt{alloca}
        instruction. Since we can get this information only from a pointer
        analysis, this field can be used only when the analysis is available as
        a plugin.
    \item \texttt{newIstruction:} Instruction that is to be inserted
    \item \texttt{in:} Name of a function, in which this rule should be
      applied. Can be set to a value "*" meaning that it should be used in all functions.
    \item \texttt{where:} Specifies the location of insertion. It can be:
      \texttt{before} or \texttt{after} the found sequence of instructions,
      \texttt{entry} (at the entry point of the given function, \texttt{in}
      cannot be set to "*" in this case) or \texttt{return} (before every
      terminator instruction of the given function, \texttt{in} cannot be set to
      "*" in this case).
    \item \texttt{setFlags:} List of pairs \texttt{<flag, value>} that sets
      all \texttt{flags} to a corresponding \texttt{value} if the rule was
      applied. This field is optional.
  \end{itemize}
\end{itemize}

In figure~\ref{fig:json_example}, we can see an example of an
\texttt{instructionRule}. In this case, everery time the tool comes across a
\texttt{load} instruction in any function (\texttt{in} is set to "*"), the only
operands of \texttt{load} is stored to a variable \texttt{<t1>}, size of the
type that is being load is stored to a variable \texttt{<t2>} and condition
\texttt{!isValidPointer} is checked. If the conditions holds, call of a
function \texttt{\_\_INSTR\_check\_pointer} with arguments \texttt{<t1>} and
\texttt{<t2>} is inserted before the current \texttt{load} instruction.

The functions whose calls are instrumented into a code must be defined in a
file specified by the \texttt{file} field in the JSON configuration. The names
of the functions must have a prefix \texttt{\_\_INSTR\_}...
\textcolor{red}{this is not true, but they are still left from the
instrumentation}



\chapter{Memory Safety Instrumentation}\label{chap:memsafety}
\section{Basic Approach}

To check memory safety, namely absence of invalid pointer dereferences, invalid
deallocations, and memory leaks, our instrumentation inserts a code that tracks
all allocated memory blocks and all memory-manipulating operations at run-time.
For every block of the memory, we maintain a record with the address and the size
of the block. The records are stored in four lists:
\begin{itemize}
  \item \stacklist for blocks allocated on the stack
  \item \heaplist for blocks allocated on the heap
  \item \globalslist for global variables
  \item \dealloclist for blocks on the heap that were already deallocated
\end{itemize}
We keep the records in the last list only to provide more precise
error descriptions. For example, the information in this list allows
us to distinguish double free error from generic invalid deallocation,
or use-after-free from vague invalid dereference error. This list can
be removed in order to reduce memory consumption.

To maintain the records, we use the following functions:
\begin{itemize}
  \item \texttt{\_\_INSTR\_remember(address, size)}
  \\Creates a record and inserts it to the \stacklist.
  \item \texttt{\_\_INSTR\_remember\_malloc(address, size)}
  \\Creates a record and inserts it to the \heaplist.
  \item \texttt{\_\_INSTR\_remember\_global(address, size)}
  \\Creates a record and inserts it to the \globalslist.
  \item \texttt{\_\_INSTR\_free(address)}
  \\Checks whether the address refers to some block allocated on the heap,
  i.e., whether there is a record with this address in the \heaplist. If such a
  record exists, it is removed from the \heaplist\ and inserted into the
  \dealloclist. If there is a record with this address, but it is found in the
  \dealloclist\, we report double free error. In all remaining cases, an invalid
  deallocation is reported.
  \item \texttt{\_\_INSTR\_destroy\_allocas()}
  \\ Since local variables on the stack are destroyed when a function
  finishes, this function removes all relevant records from the \stacklist right
  before returning from a function.
\end{itemize}

There are also two more functions that use records to check
the safety of memory operations, but they do not modify them:
\begin{itemize}
\item \texttt{\_\_INSTR\_check\_pointer(address, n)}
  \\Checks whether it is a safe operation to dereference $n$ bytes
  starting at the given address. More precisely, it checks whether
  there is a record in the \stacklist, \heaplist or the \globalslist
  covering $n$ bytes starting at the given address.  If such a record
  is found, the check is successful. If the address is covered by a
  record stored in the \dealloclist, we report a use-after-free error
  If no record covering the address is found, or there is a record
  covering the address but not all $n$ bytes starting at this address,
  we report an invalid dereference.
% \todo{NENI PRAVDA: When dereferencing a single address, we use 1 as the
% second parameter. A larger parameter is used for instrumentation of
% functions like \texttt{memset} or \texttt{memcpy}}
\item \texttt{\_\_INSTR\_check\_leaks()}
  \\Checks whether all blocks dynamically allocated on the heap have been
  freed. In other words, if there is a record in the \heaplist, then we report
  a memory leak.
\end{itemize}

The checks mentioned in the functions above
% , namely in \texttt{\_\_INSTR\_free(address)},
% \texttt{\_\_INSTR\_check\_pointer(address, n)}, and
% \texttt{\_\_INSTR\_check\_leaks()},
are in fact implemented as assertions. Hence, an error is reported if
and only if some of these asserts can be violated, i.e.~if an error
location is reachable.

Calls to the hitherto described functions are inserted into the analyzed
program to keep track of the state of the memory.
For each global variable, the instrumentation inserts a call of
\texttt{\_\_INSTR\_remember\_global} to the beginning of \emph{main} in order
to create the corresponding record. Further, the instrumentation detects memory
handling instructions in the code and inserts calls to the corresponding
\texttt{\_\_INSTR\_*} functions above or below these instructions. When
instrumenting a memory allocation instruction, we insert the call of
\texttt{\_\_INSTR\_remember} (for allocations on the stack) or
\texttt{\_\_INSTR\_remember\_malloc} (for allocations on the heap) \emph{below}
the allocation instruction as the called function needs the address of the
allocated block. In all remaining cases, the calls are inserted \emph{above} the
corresponding instruction as we want to detect a memory handling error before
the program reaches the actual error.
Calls of \texttt{\_\_INSTR\_destroy\_allocas} are instrumented \emph{above} all
\emph{return} instructions in each function. Finally, a call to
\texttt{\_\_INSTR\_check\_leaks()} is inserted at the end of \emph{main} to
check for memory leaks.

\lstset{escapeinside={<@}{@>}, columns=fullflexible, basicstyle=\ttfamily, language=llvm, style=nasm}
\begin{figure}[t]
\begin{lstlisting}
%1 = alloca i32*, align 8
<@{\color[RGB]{0, 135, 0} \%2 = bitcast i32** \%1 to i8*}@>
<@{\color[RGB]{0, 135,0} call void @\_\_INSTR\_remember(i8* \%2, i64 8, i32 1)}@>
%3 = call i8* malloc(i64 4)
<@{\color[RGB]{0, 135,0} call void @\_\_INSTR\_remember\_malloc(i8* \%3, i64 4, i32 1)}@>
%4 = bitcast i8* %3 to i32*
<@{\color[RGB]{0, 135,0} \%5 = bitcast i32** \%1 to i8*}@>
<@{\color[RGB]{0, 135,0} call void @\_\_INSTR\_check\_pointer(i8* \%5, i64 8)}@>
store i32 %4, i32** %1, align 8
%6 = bitcast i32** %1 to i8*
<@{\color[RGB]{0, 135,0} call void @\_\_INSTR\_free(i8* \%6)}@>
call void @free(i8* %6)
<@{\color[RGB]{0, 135,0} \%7 = bitcast i32** \%1 to i8*}@>
<@{\color[RGB]{0, 135,0} call void @\_\_INSTR\_check\_pointer(i8* \%7, i64 8)}@>
%8 = load i32*, i32** %1
<@{\color[RGB]{0, 135,0} \%9 = bitcast i32** \%8 to i8*}@>
<@{\color[RGB]{0, 135,0} call void @\_\_INSTR\_check\_pointer(i8* \%9, i64 8)}@>
store i32 2, i32** %8, align 4
\end{lstlisting}
\caption{Basic instrumentation of a code with an invalid pointer
  dereference.}
\label{fig:example1}
\end{figure}

Figure~\ref{fig:example1} shows a simple code containing an invalid
pointer dereference. The code is instrumented with the
\texttt{\_\_INSTR\_*} function calls which keep track of memory
operations and check their safety. To begin with, the instrumentation
remembers every memory allocation, including stack variables. It may
seem redundant at first, but without any further analysis, we can not
say that a local variable is dereferenced e.g.~using a pointer, in
which case we need to have the record, otherwise we would report
invalid dereference.
%(in particular, dereference of non-allocated memory).
The memory allocated by the call to \texttt{malloc} is remembered by the call
to \texttt{\_\_INSTR\_remember\_malloc(p, 4)}. This memory is later freed and
this fact is recorded by the call to \texttt{\_\_INSTR\_free(p)}. Every access to the memory (every \texttt{load} or \texttt{store} instruction)
are checked using \texttt{\_\_INSTR\_check\_pointer}. The last call of this
function (i.e.~line~17) reveals
use-after free error. \todo{bitcasts}

%\texttt{\_\_INSTR\_free(p)} moves the record from \heaplist to \dealloclist.
%Finally, the code is instrumented with a call to
%\texttt{\_\_INSTR\_check\_pointer(r, 4)} which detects the use-after-free
%error.

Disadvantage of this basic approach is that it tracks all memory allocations and
instruments all dereferences. The amount of inserted function calls is
therefore usually very large and since the vast majority of the new code has
an effect on reachability of error locations, it cannot be removed by slicing.


\chapter{Evaluation}\label{chap:eval}
In this chapter, we evaluate the instrumentation for checking memory safety
that was employed in \symbiotic on benchmarks from SV-COMP~2017.

We used all the benchmarks from the official \emph{MemSafety} category along
with the benchmarks from the subcategory \emph{TerminCrafted}, which was not
included in the official SV-COMP~2017. The benchmarks from these categories are
programs in C and they can contain either no violation of memory safety, or the
following errors:
\begin{itemize}
  \item invalid memory deallocation, e.g. double free error,
  \item invalid pointer dereference,
  \item memory leaks.
\end{itemize}
There were 390 benchmarks in total, \todo{X} of them containing some violation
of memory safety, \todo{Y} of them safe.

\symbiotic first instruments the program according to one of the configurations
described in Chapter~\ref{chap:memsafety}. Since the newly inserted code marks
possible error locations, we reduced the problem to a reachability problem.
\symbiotic slices the program and then runs symbolic executor \klee that checks
the reachability of the error locations. If no error location is reachable, it
answers \emph{true}. If some error location is reachable, it answeres
\emph{false} and gives information about a type of the error (e.g.  invalid
dereference or memory leak). If \symbiotic cannot decide the reachability, it
answers \emph{unknown}  and if it rund out of time, i.e. the given time limit
was exceeded, the answer is \emph{TIMEOUT}.

All the following measurements were taken on machines with \textit{Intel(R)
Core(TM) i7-3770} CPU that run on 3.40GHz frequency and dispose of 8~GB of
memory. The memory limit was set to 8~GB. We used the utility
\emph{Benchexec}~\cite{Beyer2015} for reliable measurement of consumed
resources.

\section{Comparison of Configurations}
In this section, we compare three configurations introduced in
Chapter~\ref{chap:memsafety}: the basic approach (one-phase instrumentation
without any plugins, denoted as \emph{basic}), the enhancement with a pointer
analysis as a plugin (denoted as \emph{ePTA}) and the staged instrumentation
(denoted as \emph{staged}). We ran \symbiotic with these configurations on the
above mentioned benchmarks from SV-COMP with the CPU time limit set to 120~s.

\begin{table}[t]
\begin{tabular}{l  r   r   r}
 & basic & ePTA & staged \\
 \hline
 size before instrumentation & 155782 & 155782 & 155782 \\
 size after instrumentation  & 322018 & 192468 & 171327 \\
 \hline
 inserted calls (total)    & 166236 & 36686 & 15545 \\
 inserted calls (average)  & 427 & 94 & 39 \\
\end{tabular}
\caption{The comparison of the three configurations for the memory safety
instrumentation. Size is given by the number of instructions of a program.}
\label{tab:numbers}

\end{table}

As for the number of inserted instructions, we present the experimental results
in Table~\ref{tab:numbers}. The first part of the table shows the total size of
the benchmarks expressed with the number of LLVM instructions before and after
instrumentation for each configuration. In the second part, the cumulative and
average number of inserted \texttt{call} instructions is counted. Note that the
total size before instrumentation is 155782 instructions, which is 400
instruction on average. This means that with the basic approach, the final code
after instrumentation is of double size. Evidently, the number of inserted
instructions significantly decreases with each enhancement.

\todo{times}

\begin{table}[h]
\begin{tabular}{l  r   r   r}
 & basic & ePTA & staged \\
 \hline
 true     & 116 & 123  & \textbf{182} \\
 false    & 132 & 133  & \textbf{135} \\
 unknown  & 1   & 1    & 1 \\
 timeout  & 138 & 131  & \textbf{71} \\
\end{tabular}
\caption{The answers given by \symbiotic when using the three configurations for the memory safety
instrumentation.}
\label{tab:answers}

\end{table}

The enhancements had also a positive effect on the answers of \symbiotic as
shown in Table~\ref{tab:answers}. When using the basic approach, \symbiotic did
not manage to decide 138 benchmarks in the given time limit. With a pointer
analysis, it decided 7 more benchmarks. The most significant improvement was
achieved with the staged instrumentation: \symbiotic ran out of time only in 71
cases. The both enhancements had impact especially on the benchmarks that did not
contain any violation of memory safety.

\section{Comparison with Other Tools}


\chapter{Future Work}\label{chap:future}
In the future, we would like to implement a few more features to our
instrumentation tool. First of all, we would like to enable not only insertion
of instructions, but also their replacement. This might seem a little bit
inconvenient at first sight, since the instrumentation is not supposed to
modify the given input program, however, a user might for example want to
replace calls of a function with calls of a function with the same semantics
but different implementation enriched with some analysis functionality, e.g.
logging.

Another improvement we would like to implement is the possibility to determine
what plugin should be questioned when evaluating a query. The queries are often
intended for concrete plugin and it is therefore pointless to question all
plugins at once.

Apart from the tool itself, we would also like to come up with new
configuration files for \symbiotic. Currently, we use \clang to check for
integer overflows in programs, but in the future, we want to create our own
configuration for the instrumentation to check for the overflows. This would
also lead to an implementation of a value analysis that would be used as a plugin.
As for memory safety configuration, we want to try replacing the lists for
preserving records with more efficient structure, such as trees.


\chapter{Conclusion}\label{chap:conclusion}
Within this thesis, we listed tools that perform instrumentation of LLVM~IR. We
designed and implemented a new tool for general instrumentation of LLVM IR such
that it can be easily configured with two files provided by a user. Moreover,
the tool can use results of static analyzers to reduce the amount of the
inserted code.

We also created a~configuration for this tool that is aimed at checking memory
safety of programs. We improved this configuration with two enhancements:
(i)~we employed a~pointer analysis to find out if it can itself decide
whether a pointer dereference is valid, and (ii)~we divided the instrumentation
into stages to omit some of the memory allocations from the instrumentation.
Both of these enhancements led to a smaller amount of inserted code and
therefore to a~faster analysis of the code.

Finally, we evaluated the configurations for checking memory safety by
employing the instrumentation in the tool \symbiotic and running it on a set of
benchmarks from SV-COMP. The enhanced configurations proved to be of great
benefit as \symbiotic won the first place in category \emph{MemSafety}
in SV-COMP~2018.


\addcontentsline{toc}{chapter}{Bibliography}
\printbibliography

\appendix %% Start the appendices.
\chapter{An appendix}
Here you can insert the appendices of your thesis.


\end{document}


LLVM~\cite{llvm} is an open source project that provides compiler technologies
designed to be independent of a programming language and a~target architecture.
The project has several parts: LLVM intermediate representation (LLVM IR, often
shortened to LLVM) for representation of a source code, LLVM Core libraries for
code generation, optimization and transformation, and tools such as Clang
compiler, LLVM linker \texttt{llvm-link} or LLVM optimizer \texttt{opt}.
Another important part of LLVM project is LLVM Pass Framework, since LLVM
passes are programs used for transformations and optimizations of LLVM code. In
this chapter, we explain the basics of LLVM IR, which can be used in three
different forms: human readable representation, bitcode representation, and an
in-memory compiler representation. For the sake of simplicity, we will work
only with the human readable form as all the forms are equal.


\section{LLVM structure}

The high-level structure of an LLVM program consists of modules. Each module
corresponds to one compilation unit. More modules can be linked together into
one module with the LLVM linker. They contain global variables and functions.

Each function begins with the \texttt{define} keyword and is composed of basic
blocks. Basic blocks are sequences of instructions, that have a single entry node
and a single exit node and there is no branching in them. Basic blocks form
a control flow graph for a function. In Figure~\ref{fig:llvm_example} we give
an example of an LLVM module with two function definitions, each containing one
basic block.

LLVM is in the static single assignment form~\cite{ssa} (SSA form), which means that each
variable can be assigned only at one location in the code. It uses:

\begin{itemize}
    \item global identifiers that begin with the '@' character for functions
    and global variables, for example the global variable \texttt{@number} in
    Figure~\ref{fig:llvm_example}
    \item local identifiers that begin with the '\%' character for register
    names and types, for example the local variable \texttt{\%a} in
    Figure~\ref{fig:llvm_example}
\end{itemize}

Variables can be named or unnamed. Identifiers for unnamed variables are
represented as unsigned numbers with a corresponding prefix (e.g.~local
identifier \texttt{\%1} in Figure~\ref{fig:llvm_example}).

\begin{figure}[t]
 \lstinputlisting[language=llvm,style=nasm]{examples/llvm.ll}
 \caption{An example of an LLVM module with function \texttt{main} that calls
 function \texttt{foo} which allocates an array of ten integers and stores the
 value of the global variable \texttt{number} in the first field of the array.}
 \label{fig:llvm_example}
\end{figure}

\section{LLVM types}

LLVM is a strongly typed language. There are simple types such as integer type
(e.g.~\texttt{i32} for a 32-bit integer), floating point types
(e.g.~\texttt{float} or \texttt{double}) and pointer type (e.g.~\texttt{i32*}
for a pointer to a 32-bit integer), and also types for vectors
(e.g.~\texttt{<10 x i32>} for a vector of ten 32-bit integers), arrays
(e.g.~\texttt{[10 x i32]} for an array of ten 32-bit integers) and
structures (e.g.~\texttt{\{i32, i32\}} for a pair of 32-bit integers).
Observe that in Figure~\ref{fig:llvm_example} the global variable
\texttt{number} is a 32-bit integer and \texttt{a} is an array of ten
32-bit integers.

\section{Basic instructions}

In this section, we describe the most important instructions for this thesis. The
complete list of instructions can be found in the LLVM
documentation\footnote{\url{https://llvm.org/docs/LangRef.html}}.

A terminator instruction is used as the last instruction of each basic block
and it determines how the control flow will continue. For our purpose we will
describe only two of the terminator instructions: \texttt{ret} and \texttt{br}.

\begin{description}
\item[\texttt{ret}] is used to return from a function to a basic block
from which the function was called. It has one optional argument that represent
a return value of the function. In Figure~\ref{fig:llvm_example} we can see the
two variants of this instruction: in function \texttt{foo}, there is a
\texttt{ret void} instruction, because this function does not return any value,
whereas function \texttt{main} returns 0 (\texttt{ret i32 0}).

\item[\texttt{br}] determines which basic block from the current function
will follow. It represents either conditional branching:

\texttt{br i1 <condition>, label <true>, label <false>},

transferring the control flow to \texttt{true} block if the \texttt{condition} holds and to
\texttt{false} block otherwise, or unconditional branching:

\texttt{br label <b>},

transferring the control flow to the block \texttt{b} unconditionally.
\end{description}

The most important instructions for working with memory are \texttt{load},
\texttt{store}, \texttt{alloca} and \texttt{getelementptr}:
\begin{description}
\item[\texttt{load}] is used to read from memory specified by its operand.
\item[\texttt{store}] is used to write to memory and has two operands: a value
to store and the address of a target memory.
\item[\texttt{alloca}] allocates memory on the stack.
\item[\texttt{getelementptr}] gets the address of some element of an aggregate.
It can be used for example to get the address of an element of an array.
\end{description}
We can see the usage of these instructions in Figure~\ref{fig:llvm_example} in
the function \texttt{foo}. The \texttt{alloca} instruction on line~4 allocates an
array \texttt{arr} of ten 32-bit integers on the stack. The \texttt{load}
instruction on line~5 reads the value of the global variable \texttt{number} and
names it as \texttt{\%1}. The value is later stored to \texttt{\%2} on line~8.
\texttt{\%2} is the address of the fifth element of the array~\texttt{arr} since the
\texttt{getelementptr} instruction was used on line~6 on the array \texttt{arr}
with the last operand equal to 5.

Other relevant instructions are \texttt{call} and \texttt{bitcast}:
\begin{description}
\item[\texttt{call}] instruction calls a function given as its
operand together with function's arguments.
\item[\texttt{bitcast}] converts a given value to a given type, e.g.~\texttt{bitcast i8 1 to
i32} converts 8-bit integer of value 1 to 32-bit integer of value 1.
\end{description}
 We can find an example of a \texttt{call} instruction in
 Figure~\ref{fig:llvm_example} in the function \texttt{main} which calls
 the function \texttt{foo} on line~15. This instruction can also be used for
 dynamic memory allocation by calling a function for dynamic allocation, e.g.
 \texttt{malloc}.

Binary operator instructions have two operands of the same type and return a
result of an operation on these operands, for example \texttt{add} instruction
for addition or \texttt{sub} instruction for subtraction. There are usually two
versions of these instructions: one for integer values and one for floating
point values.

Bitwise binary instructions are used for bitwise operations. This category
contains for example bitwise logical operators like \texttt{and} instruction,
\texttt{or} instruction, etc.



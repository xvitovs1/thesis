\url{www.aosabook.org/en/llvm.html}
\medskip

LLVM is an open source project that provides compiler technologies designed to
be independent of a target architecture. It uses LLVM IR as an intermediate
representation code. It can be used in three different forms: human readable
representation, bitcode representation and an in-memory compiler IR. From this
point we will take only the human readable form into consideration for the sake
of simplicity.

\begin{figure}[h]
 \lstinputlisting[language=llvm,style=nasm]{examples/llvm.ll}
 \caption{Example of an LLVM module with function \texttt{main} that calls
 \texttt{foo} function which allocates an array of ten integers and stores
 \texttt{number} in the first field of the array.}
 \label{fig:llvm_example}
\end{figure}

\section{LLVM structure} %TODO rename the chapter to LLVM IR?

The high-level structure of an LLVM program consists of modules, which are
units created when a program is translated. More modules can be linked together
with the LLVM linker. These units contain global variables and functions.

Each function begins with a \texttt{define} keyword and is composed of basic
blocks. Basic blocks are sequences of instructions and have a single entry node
and a single exit node, i.e. there is no branching in a basic block. They form
a control flow graph for a function. In figure~\ref{fig:llvm_example} we give
an example of an LLVM module with two function definitions, each containing one
basic block.

LLVM is in SSA form (static single assignment form), which means that each
variable can be assigned only once and must be defined before its use. It uses:

\begin{itemize}
    \item global identifiers that begin with the '@' character for functions
    and global variables, for example global variable \texttt{@number} in
    figure~\ref{fig:llvm_example}, and
    \item local identifiers that begin with the '\%' character for register
    names and types, for example a local variable \texttt{\%a} in
    figure~\ref{fig:llvm_example}.
\end{itemize}

Both kinds of identifiers can be named or unnamed, unnamed identifiers are
represented as unsigned numeric values (e.g. local identifier \texttt{\%1} in
figure~\ref{fig:llvm_example}).

\section{LLVM types}

LLVM is a strongly typed language, i.e. all values and variables are typed.
There are simple types such as integer type (e.g. \texttt{i32} for a 32-bit
integer), floating point types (e.g. \texttt{float} or \texttt{double}) and
pointer type (e.g. \texttt{i32*} for a pointer to a 32-bit integer), and also
types for vectors (e.g. \texttt{<10 x i32>} for a vector of 10 32-bit
integers), arrays (e.g. \texttt{[10 x i32]} for an array of 10 32-bit integers)
and structures (e.g. \texttt{\{i32, i32\}} for a pair of 32-bit integers). For
example, in figure~\ref{fig:llvm_example} we can see that global variable
\texttt{number} is a 32-bit integer and \texttt{a} is an array of ten
32-bit integers.

\section{Basic instructions}

There are five classes of LLVM instructions: terminator instructions, binary
instructions, bitwise binary instructions, memory instructions and other
instructions.

A terminator instruction is used as the last instruction of each basic block
and it determines which block will follow after the current one. For our
purposes we will describe only two of the terminator instructions: \texttt(ret)
and \texttt{br}.

\texttt{ret} instruction is used to return from a function to a basic block
from which the function was called. It has one optional argument that represent
a return value of a function. In figure~\ref{fig:llvm_example} we can see the
two variants of this instruction: in function \texttt{foo}, there is a
\texttt{ret void} instruction, because this function does not return any value,
whereas function \texttt{main} returns 0 (\texttt{ret i32 0}).

\texttt{br} instruction determines which basic block from the current function
will follow. It represents either conditional branching \texttt{br i1
<condition>, label <true-branch>, label <false-branch>} transfering the control
flow to \texttt{true-branch} block if the \texttt{condition} holds and to
\texttt{false-branch} block otherwise, or uncoditional branching \texttt{br
label <b>} transfering the control flow to a block \texttt{b} unconditionally.

Binary operator instructions have two operands of the same type and return a
result of an operation on these operands, for example \texttt{add} instruction
for addition or \texttt{sub} instruction for subtraction. There are usually two
versions of these instructions: one for integer values and one for floating
point values. These instructions together with bitwise binary instructions
that are used for bitwise operations are not relevant for this work.

The two most important instructions for working with memory are \texttt{load}
and \texttt{store}. \texttt{load} instruction is used to read from memory
specified by its operand, whether \texttt{store} is used to write to memory and
has two operands: a value to store and address of a target memory. We can see
the usage of \texttt{load} and \texttt{store} in figure~\ref{fig:llvm_example}
in function \texttt{foo} where a value of global variable \texttt{number} is
read, marked as \texttt{\%1} and later stored to \texttt{\%2}. Another
instruction relevant for our work is \texttt{alloca} instruction for allocating
memory on the stack. In figure~\ref{fig:llvm_example}, in function \texttt{foo}
\texttt{alloca} instruction is used to allocate \texttt{a} as an array of ten
32-bit integers on the stack. Instruction \texttt{getelementptr} gets the
address of some element of an aggregate data structure, for example in
figure~\ref{fig:llvm_example} in function \texttt{foo} it gets the address of
the fifth element of array \texttt{a}.

Other relevant instructions are \texttt{call} instruction and \texttt{bitcast}
instruction. \texttt{call} instruction calls calls a function given as its
operand together with function's arguments. We can find an example of a
\texttt{call} instruction in figure~\ref{fig:llvm_example} in function
\texttt{main} which calls function \texttt{foo} on line~15. \texttt{bitcast}
instruction converts given value to a given type, e.g. \texttt{bitcast i8 1 to
i32} converts 8-bit integer of value 1 to 32-bit integer of value 1.

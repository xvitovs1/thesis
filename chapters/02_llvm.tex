\url{www.aosabook.org/en/llvm.html}
\medskip

LLVM is an open source project that provides compiler technologies designed to
be independent of a target architecture. It uses LLVM IR as an intermediate
representation code. It can be used in three different forms: human readable
representation, bitcode representation and an in-memory compiler IR. From this
point we will take only the human readable form into consideration.

\section{LLVM IR structure} %TODO rename the chapter to LLVM IR?

LLVM IR is in SSA form (static single assignment form), which means that each
variable can be assigned only once and must be defined before its use. It uses:

\begin{itemize}
    \item global identifiers that begin with the '@' character for functions
and global variables and
    \item local identifiers that begin with the '\%' character for register
names and types, for example local variable $b$.
\end{itemize}

Both kinds of identifiers can be named or unnamed, unnamed identifiers are
represented as an unsigned numeric values. 

The high-level structure of an LLVM IR program consists of modules, which are
units created when a program is translated. More modules can be linked together
with the LLVM linker. Modules contain global variables and functions.

Each function begins with a \texttt{define} keyword and is composed of basic
blocks. Basic blocks are sequences of instructions and have a single entry node
and a single exit node, i.e. there is no branching in a basic block. They form
a control flow graph for a function. 


\textcolor{red}{TODO example of a module with description}

\section{LLVM IR types}

LLVM IR is a strongly typed language.

\section{Basic instructions}


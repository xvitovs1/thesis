LLVM~\cite{llvm} is an open source project that provides compiler technologies
designed to be independent of a target architecture. It uses LLVM IR as an
intermediate representation code. It can be used in three different forms:
human readable representation, bitcode representation and an in-memory compiler
IR. From this point on we will take only the human readable form into
consideration for the sake of simplicity.

\begin{figure}[h]
 \lstinputlisting[language=llvm,style=nasm]{examples/llvm.ll}
 \caption{An example of an LLVM module with function \texttt{main} that calls
 function \texttt{foo} which allocates an array of ten integers and stores
 \texttt{number} in the first field of the array.}
 \label{fig:llvm_example}
\end{figure}

\section{LLVM structure} %TODO rename the chapter to LLVM IR?

The high-level structure of an LLVM program consists of modules, which are
units created when a program is translated. More modules can be linked together
into one module with the LLVM linker. These units contain global variables and
functions.

Each function begins with a \texttt{define} keyword and is composed of basic
blocks. Basic blocks are sequences of instructions, they have a single entry node
and a single exit node and there is no branching in them. They form
a control flow graph for a function. In Figure~\ref{fig:llvm_example} we give
an example of an LLVM module with two function definitions, each containing one
basic block.

LLVM is in static single assignment form~\cite{ssa} (SSA form), which means that each
variable can be assigned only at one location in teh code. It uses:

\begin{itemize}
    \item global identifiers that begin with the '@' character for functions
    and global variables, for example global variable \texttt{@number} in
    Figure~\ref{fig:llvm_example}, and
    \item local identifiers that begin with the '\%' character for register
    names and types, for example a local variable \texttt{\%a} in
    Figure~\ref{fig:llvm_example}.
\end{itemize}

Both kinds of identifiers can be named or unnamed, unnamed identifiers are
represented as unsigned numeric values (e.g. local identifier \texttt{\%1} in
Figure~\ref{fig:llvm_example}).

\section{LLVM types}

LLVM is a strongly typed language. There are simple types such as integer type
(e.g. \texttt{i32} for a 32-bit integer), floating point types (e.g.
\texttt{float} or \texttt{double}) and pointer type (e.g. \texttt{i32*} for a
pointer to a 32-bit integer), and also types for vectors (e.g. \texttt{<10 x
i32>} for a vector of ten 32-bit integers), arrays (e.g. \texttt{[10 x i32]}
for an array of ten 32-bit integers) and structures (e.g. \texttt{\{i32, i32\}}
for a pair of 32-bit integers). Observe that in Figure~\ref{fig:llvm_example}
the global variable \texttt{number} is a 32-bit integer and
\texttt{a} is an array of ten 32-bit integers.

\section{Basic instructions}

There are five classes of LLVM instructions: terminator instructions, binary
instructions, bitwise binary instructions, memory instructions and other
instructions. \todo{In this section, we describe the most important
instructions for this work. The complete list of instructions can be found in
LLVM documentation\footnote{\url{https://llvm.org/docs/LangRef.html}}.}

A terminator instruction is used as the last instruction of each basic block
and it determines how the control flow will continue. For our purpose we will
describe only two of the terminator instructions: \texttt{ret} and \texttt{br}.

\begin{description}
\item[\texttt{ret}] is used to return from a function to a basic block
from which the function was called. It has one optional argument that represent
a return value of a function. In Figure~\ref{fig:llvm_example} we can see the
two variants of this instruction: in function \texttt{foo}, there is a
\texttt{ret void} instruction, because this function does not return any value,
whereas function \texttt{main} returns 0 (\texttt{ret i32 0}).

\item[\texttt{br}] determines which basic block from the current function
will follow. It represents either conditional branching:

\texttt{br i1 <condition>, label <true>, label <false>},

transfering the control flow to \texttt{true} block if the \texttt{condition} holds and to
\texttt{false} block otherwise, or uncoditional branching:

\texttt{br label <b>},

transfering the control flow to the block \texttt{b} unconditionally.
\end{description}

Binary operator instructions have two operands of the same type and return a
result of an operation on these operands, for example \texttt{add} instruction
for addition or \texttt{sub} instruction for subtraction. There are usually two
versions of these instructions: one for integer values and one for floating
point values.

Bitwise binary instructions are used for bitwise operations. This category
contains for example bitwise logical operators like \texttt{and} instruction,
\texttt{or} instruction, etc.

The two most important instructions for working with memory are \texttt{load}
and \texttt{store}:
\begin{description}
\item[\texttt{load}] is used to read from memory specified by its operand,
\item[\texttt{store}] is used to write to memory and has two operands: a value
to store and address of a target memory.
\end{description}
We can see the usage of \texttt{load} and \texttt{store} in
Figure~\ref{fig:llvm_example} in function \texttt{foo} where a value of global
variable \texttt{number} is read, named as \texttt{\%1} and later stored to
\texttt{\%2}. Another instruction important for our work is \texttt{alloca}
instruction:
\begin{description}
\item[\texttt{alloca}] allocates memory on the stack.
\end{description}
In Figure~\ref{fig:llvm_example} \texttt{alloca} instruction is used to
allocate \texttt{a} as an array of ten 32-bit integers on the stack in function
\texttt{foo}. Instruction \texttt{getelementptr} gets the address of some
element of an aggregate data structure, for example in
Figure~\ref{fig:llvm_example} in function \texttt{foo} it gets the address of
the fifth element of array \texttt{a}.

Other relevant instructions are \texttt{call} instruction and \texttt{bitcast}
instruction:
\begin{description}
\item[\texttt{call}] instruction calls a function given as its
operand together with function's arguments,
\item[\texttt{bitcast}] converts a given value to a given type, e.g.~\texttt{bitcast i8 1 to
i32} converts 8-bit integer of value 1 to 32-bit integer of value 1.
\end{description}
 We can find an example of a \texttt{call} instruction in
 Figure~\ref{fig:llvm_example} in function \texttt{main} which calls function
 \texttt{foo} on line~15.

\url{www.aosabook.org/en/llvm.html}
\medskip

LLVM is an open source project that provides compiler technologies designed to
be independent of a target architecture. It uses LLVM IR as an intermediate
representation code. It can be used in three different forms: human readable
representation, bitcode representation and an in-memory compiler IR. From this
point we will take only the human readable form into consideration.

\section{LLVM IR structure} %TODO rename the chapter to LLVM IR?

LLVM IR is in SSA form (static single assignment form), which means that each
variable can be assigned only once and must be defined before its use. It uses:

\begin{itemize}
    \item global identifiers that begin with the '@' character for functions
and global variables and
    \item local identifiers that begin with the '\%' character for register
names and types, for example local variable $b$.
\end{itemize}

Both kinds of identifiers can be named or unnamed, unnamed identifiers are
represented as unsigned numeric values. 

The high-level structure of an LLVM IR program consists of modules, which are
units created when a program is translated. More modules can be linked together
with the LLVM linker. These units contain global variables and functions.

Each function begins with a \texttt{define} keyword and is composed of basic
blocks. Basic blocks are sequences of instructions and have a single entry node
and a single exit node, i.e. there is no branching in a basic block. They form
a control flow graph for a function. 

\textcolor{red}{TODO example of a module with description}

\section{LLVM IR types}

LLVM IR is a strongly typed language, i.e. all values and variables are typed.
There are simple types such as integer type (e.g. \texttt{i32} for a 32-bit
integer), floating point types (e.g. \texttt{float} or \texttt{double}) and
pointer type (e.g. \texttt{i32*} for a pointer to a 32-bit integer), and also
types for vectors (e.g. \texttt{<10 x i32>} for a vector of 10 32-bit
integers), arrays (e.g. \texttt{[10 x i32]} for an array of 10 32-bit integers)
and structures (e.g. \texttt{\{i32, i32\}} for a pair of 32-bit integers).

\section{Basic instructions}

There are five classes of LLVM instructions:

\begin{enumerate}
 \item terminator instructions,
 \item binary instructions,
 \item bitwise binary instructions,
 \item memory instructions and
 \item other instructions.
\end{enumerate}

A terminator instruction is used as the last instruction of each basic block
and it determines which block will follow after the current one. For our
purposes we will describe only two of the terminator instructions: \texttt(ret)
and \texttt{br}.

\texttt{ret} instruction is used to return from a function to a basic block
from which the function was called. It has one optional argument that represent
a return value of a function. The syntax of the \texttt{ret} instruction is
following:

\medskip
\noindent \texttt{ret <type> <value>} 

\noindent \texttt{ret void}
\medskip

\texttt{br} instruction determines which basic block from the current function
will follow. It represents either conditional branching \texttt{br i1
<condition>, label <true-branch>, label <false-branch>} transfering the control
flow to \texttt{true-branch} block if the \texttt{condition} holds and to
\texttt{false-branch} block otherwise, or uncoditional branching \texttt{br
label <b>} transfering the control flow to a block \texttt{b} unconditionally.

Binary operator instructions have two operands of the same type and return a
result of an operation on these operands, for example \texttt{add} instruction
for addition or \texttt{sub} instruction for subtraction. There are usually two
    versions of these instructions: one for integer values and one for floating
    point values.

Bitwise binary instructions \textcolor{red}{TODO}

Memory instruction \textcolor{red}{TODO}.

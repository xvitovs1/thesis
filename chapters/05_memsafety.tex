\section{Basic Approach}

To check memory safety, namely absence of invalid pointer dereferences, invalid
deallocations, and memory leaks, our instrumentation inserts a code that tracks
all allocated memory blocks and all memory-manipulating operations at run-time.
For every block of the memory, we maintain a record with the address and the size
of the block. The records are stored in four lists:
\begin{itemize}
  \item \stacklist for blocks allocated on the stack
  \item \heaplist for blocks allocated on the heap
  \item \globalslist for global variables
  \item \dealloclist for blocks on the heap that were already deallocated
\end{itemize}
We keep the records in the last list only to provide more precise
error descriptions. For example, the information in this list allows
us to distinguish double free error from generic invalid deallocation,
or use-after-free from vague invalid dereference error. This list can
be removed in order to reduce memory consumption.

To maintain the records, we use the following functions:
\begin{itemize}
  \item \texttt{\_\_INSTR\_remember(address, size)}
  \\Creates a record and inserts it to the \stacklist.
  \item \texttt{\_\_INSTR\_remember\_malloc(address, size)}
  \\Creates a record and inserts it to the \heaplist.
  \item \texttt{\_\_INSTR\_remember\_global(address, size)}
  \\Creates a record and inserts it to the \globalslist.
  \item \texttt{\_\_INSTR\_free(address)}
  \\Checks whether the address refers to some block allocated on the heap,
  i.e., whether there is a record with this address in the \heaplist. If such a
  record exists, it is removed from the \heaplist\ and inserted into the
  \dealloclist. If there is a record with this address, but it is found in the
  \dealloclist\, we report double free error. In all remaining cases, an invalid
  deallocation is reported.
  \item \texttt{\_\_INSTR\_destroy\_allocas()}
  \\ Since local variables on the stack are destroyed when a function
  finishes, this function removes all relevant records from the \stacklist right
  before returning from a function.
\end{itemize}

There are also two more functions that use records to check
the safety of memory operations, but they do not modify them:
\begin{itemize}
\item \texttt{\_\_INSTR\_check\_pointer(address, n)}
  \\Checks whether it is a safe operation to dereference $n$ bytes
  starting at the given address. More precisely, it checks whether
  there is a record in the \stacklist, \heaplist or the \globalslist
  covering $n$ bytes starting at the given address.  If such a record
  is found, the check is successful. If the address is covered by a
  record stored in the \dealloclist, we report a use-after-free error
  If no record covering the address is found, or there is a record
  covering the address but not all $n$ bytes starting at this address,
  we report an invalid dereference.
% \todo{NENI PRAVDA: When dereferencing a single address, we use 1 as the
% second parameter. A larger parameter is used for instrumentation of
% functions like \texttt{memset} or \texttt{memcpy}}
\item \texttt{\_\_INSTR\_check\_leaks()}
  \\Checks whether all blocks dynamically allocated on the heap have been
  freed. In other words, if there is a record in the \heaplist, then we report
  a memory leak.
\end{itemize}

The checks mentioned in the functions above
% , namely in \texttt{\_\_INSTR\_free(address)},
% \texttt{\_\_INSTR\_check\_pointer(address, n)}, and
% \texttt{\_\_INSTR\_check\_leaks()},
are in fact implemented as assertions. Hence, an error is reported if
and only if some of these asserts can be violated, i.e.~if an error
location is reachable.

Calls to the hitherto described functions are inserted into the analyzed
program to keep track of the state of the memory.
For each global variable, the instrumentation inserts a call of
\texttt{\_\_INSTR\_remember\_global} to the beginning of \emph{main} in order
to create the corresponding record. Further, the instrumentation detects memory
handling instructions in the code and inserts calls to the corresponding
\texttt{\_\_INSTR\_*} functions above or below these instructions. When
instrumenting a memory allocation instructioni (e.g. \texttt{alloca}), we insert the call of
\texttt{\_\_INSTR\_remember} (for allocations on the stack) or
\texttt{\_\_INSTR\_remember\_malloc} (for allocations on the heap) \emph{below}
the allocation instruction as the called function needs the address of the
allocated block. In all remaining cases, the calls are inserted \emph{above} the
corresponding instruction as we want to detect a memory handling error before
the program reaches the actual error.
Calls of \texttt{\_\_INSTR\_destroy\_allocas} are instrumented \emph{above} all
return instructions in each function. Finally, a call to
\texttt{\_\_INSTR\_check\_leaks()} is inserted at the end of \emph{main} to
check for memory leaks.

\lstset{escapeinside={<@}{@>}, columns=fullflexible, basicstyle=\ttfamily, language=llvm, style=nasm}
\begin{figure}[t]
\begin{lstlisting}
%1 = alloca i32*, align 8
<@{\color[RGB]{0, 135, 0} \%2 = bitcast i32** \%1 to i8*}@>
<@{\color[RGB]{0, 135,0} call void @\_\_INSTR\_remember(i8* \%2, i64 8, i32 1)}@>
%3 = call i8* malloc(i64 4)
<@{\color[RGB]{0, 135,0} call void @\_\_INSTR\_remember\_malloc(i8* \%3, i64 4, i32 1)}@>
%4 = bitcast i8* %3 to i32*
<@{\color[RGB]{0, 135,0} \%5 = bitcast i32** \%1 to i8*}@>
<@{\color[RGB]{0, 135,0} call void @\_\_INSTR\_check\_pointer(i8* \%5, i64 8)}@>
store i32 %4, i32** %1, align 8
%6 = bitcast i32** %1 to i8*
<@{\color[RGB]{0, 135,0} call void @\_\_INSTR\_free(i8* \%6)}@>
call void @free(i8* %6)
<@{\color[RGB]{0, 135,0} \%7 = bitcast i32** \%1 to i8*}@>
<@{\color[RGB]{0, 135,0} call void @\_\_INSTR\_check\_pointer(i8* \%7, i64 8)}@>
%8 = load i32*, i32** %1
<@{\color[RGB]{0, 135,0} \%9 = bitcast i32** \%8 to i8*}@>
<@{\color[RGB]{0, 135,0} call void @\_\_INSTR\_check\_pointer(i8* \%9, i64 8)}@>
store i32 2, i32** %8, align 4
\end{lstlisting}
\caption{Basic instrumentation of a code with an invalid pointer
  dereference.}
\label{fig:example1}
\end{figure}

Figure~\ref{fig:example1} shows a simple code containing an invalid
pointer dereference. The code is instrumented with the
\texttt{\_\_INSTR\_*} function calls which keep track of memory
operations and check their safety. To begin with, the instrumentation
remembers every memory allocation, including stack variables. It may
seem redundant at first, but without any further analysis, we can not
say that a local variable is dereferenced e.g.~using a pointer, in
which case we need to have the record, otherwise we would report
invalid dereference.
%(in particular, dereference of non-allocated memory).
The memory allocated by the call to \texttt{malloc} is remembered by the call
to \texttt{\_\_INSTR\_remember\_malloc(p, 4)}. This memory is later freed and
this fact is recorded by the call to \texttt{\_\_INSTR\_free(p)}. Every access to the memory (every \texttt{load} or \texttt{store} instruction)
are checked using \texttt{\_\_INSTR\_check\_pointer}. The last call of this
function (i.e.~line~17) reveals
use-after free error. \todo{bitcasts}

%\texttt{\_\_INSTR\_free(p)} moves the record from \heaplist to \dealloclist.
%Finally, the code is instrumented with a call to
%\texttt{\_\_INSTR\_check\_pointer(r, 4)} which detects the use-after-free
%error.

Disadvantage of this basic approach is that it tracks all memory allocations and
instruments all dereferences. The amount of inserted function calls is
therefore usually very large and since the vast majority of the new code has
an effect on reachability of error locations, it cannot be removed by slicing.

\section{Extension with Pointer Analysis}\label{sec:pta}

\begin{figure}[t]
\begin{lstlisting}[language=C]
int *r, *p;
<@\vdots@>
<@{\color{blue} // r may point to NULL or p}@>
<@{\color[RGB]{0, 135,0} // r may point to NULL}@>
r = NULL;
<@\vdots@>
<@{\color{blue} // r may point to NULL or p}@>
<@{\color[RGB]{0, 135,0} // r may point to NULL}@>
p = malloc(4);
<@\vdots@>
<@{\color{blue} // r may point to NULL or p}@>
<@{\color[RGB]{0, 135,0} // r may point to NULL or p}@>
r = p;
<@\vdots@>
<@{\color{blue} // r may point to NULL or p}@>
<@{\color[RGB]{0, 135,0} // r may point to NULL or p }@><@{\color{red} or is invalidated}@>
free(p);
<@\vdots@>
<@{\color{blue} // r may point to NULL or p}@>
<@{\color[RGB]{0, 135,0} // r may point to NULL or p }@><@{\color{red} or is invalidated}@>
*r = 1;
\end{lstlisting}
\caption{The difference between \emph{flow-insensitive} (blue), \emph{flow-sensitive} (green) pointer analysis and \emph{extended flow-sensitive} pointer analysis (red). For the sake of simplicity, the code is in C and not LLVM.}
\label{fig:example2}
\end{figure}


Blindly instrumenting every manipulation with memory brings a big
overhead in the reachability analysis because of the number of added
calls that track or check the state of memory. This problem can be
mitigated using pointer analysis as a plugin.

For every pointer in the program, pointer analysis answers the
question "What memory locations may be referenced by the pointer?"
Thus the result of pointer analysis is a set of memory locations (the
so-called points-to set) for every pointer. For the sake of
soundness, a points-to set can possibly contain the \emph{unknown}
entry where the analysis can not establish any information about
referenced memory locations. The precision of pointer analysis can be
tuned in several directions. One of the important traits of pointer
analysis is whether it is \emph{flow-sensitive} or
\emph{flow-insensitive}~\cite{Hind01}, meaning whether it takes into
consideration the flow of data in the program and computes points-to
sets for every location in the program separately (flow-sensitive), or
whether it ignores the execution order of instructions and computes
summary information about a pointer that holds at any location in the
program (flow-insensitive). For instance, in
Figure~\ref{fig:example2} a flow-insensitive analysis would tell us
that \emph{r} may point to either NULL or \emph{p} anywhere in the
program. The flow-sensitive analysis can give us information that
until the \texttt{r = p;} assignment is performed, \emph{r} points to
\texttt{NULL} and after that it points to \emph{p}. Another important
characteristic of pointer analysis is whether it is
\emph{field-sensitive}, that is whether it differentiates between
elements of aggregate objects (arrays, structures, ...) or whether it
takes an aggregate object as a single object without any sub-parts
(\emph{field-insensitive}).

With information from pointer analysis, we can say which memory may be
dereferenced in many cases. If every possible memory dereference done
by an instruction is safe, we do not need to insert the check for
invalid dereference above this instruction. For example, in
Figure~\ref{fig:example2} the assignments to variables \emph{p} and
\emph{r} can be proved to be always safe and thus need not to be
instrumented with a check. Nevertheless, typical pointer analyses do
not take into account information whether the memory was freed or
whether the lifetime of a local variable has ended because of the end
of its scope.  This would cause problems in our approach in cases like
the one in Figure~\ref{fig:example2} where the assignment \texttt{*r =
  1} dereferences already freed memory.  Without information about
invalidated memory, pointer analysis tells us that the assignment
dereferences the memory allocated by the call to \texttt{malloc(4)}
and thus writing 4 bytes to that memory is a safe operation.
Instrumentation would incorrectly omit the check in this case.

There exist sophisticated forms of pointer analysis (e.g. shape
analysis~\cite{Rinetzky2001,Hind01}) that can model the heap and the
stack and track the state of memory.  However, these analyses are too
complex for our use case. We do not want the pointer analysis model
the heap and the stack, because this is done by the reachability
analysis later.  Therefore we extended a simple flow-sensitive
Andersen's style~\cite{and94} pointer analysis so that it can track
whether a memory block was possibly invalidated (i.e.~it was freed or
its lifetime ended).  This \emph{extended pointer analysis} must be
flow-sensitive in order to get reasonable results, as flow-insensitive
analysis would mark all memory blocks that were freed somewhere in the
program as possibly invalidated (freed).

% The predicate that is of a great importance for checking memory
% safety is \emph{mayBeInvalidPointer(addr, len)}.
The query that we use for checking memory safety is
\texttt{mayBeInvalidPointer(addr, len)}. \todo{rewrite this when conditions are updated} Satisfiability of this
query is determined through the extended pointer
analysis, which returns that this query holds if dereferencing
\texttt{len} bytes starting from the address given in \texttt{addr}
argument may be unsafe or when the analysis does not have enough
information to refute the validity of the predicate.  The reader could
notice that we do not pass the location in the program to the
predicate even though we claimed that the extended pointer analysis is
flow-sensitive. This is because \llvm that we use as the intermediate
representation is in partial-SSA form~\cite{Lattner04}, so the pointer
analysis can infer the location from the \texttt{addr} variable.
%\todo{Here we use that fact that LLVM is in partial-SSA, otherwise we must pass also location
%in the program to that predicate...}
We added the condition \texttt{mayBeInvalidPointer(addr, len)} to
every rule that instruments reading or writing from/to memory (where
\texttt{addr} is the address used to access the memory and
\texttt{len} is the number of read/written bytes). \todo{example}

\section{Extension with Staged Instrumentation}\label{sec:staged}
% \begin{figure}[t]
% 	\begin{lstlisting}
%     int array[10];
%     <@{\color[RGB]{0, 135,0} \_\_INSTR\_remember(array, 10*4);}@>
%     <@\vdots@>
%     <@{\color{gray} // Pointer analysis evaluated this dereference as safe,}@>
%     <@{\color{gray} // it will not be instrumented with a check.}@>
%     array[0] = 1; <@{\color{gray} // the only dereference of array}@>
% 	\end{lstlisting}
% 	\caption{Instrumentation of a code with a safe dereference.}
% 	\label{fig:no_check_example}
% \end{figure}
%
Although the pointer analysis helps us to eliminate checks of safe
dereferences, the approach still tracks all memory allocations.
Now, there can be no corresponding checks for many of the remembered records.
Keeping such records is useless as we will never try to look them up in any of the lists.
Moreover, if fewer records are stored, the whole verification process is faster
because smaller lists are searched.
% Figure~\ref{fig:no_check_example} shows a
% scenario where the allocation of \texttt{array} is instrumented with a call to
% \texttt{\_\_INSTR\_remember} even though pointer analysis evaluated the only
% dereference of \texttt{array} as safe and no check was inserted. In this case,
% we do not need to remember the record since the information will not be used
% anywhere.
Figure~\ref{fig:example4} on the right shows a scenario where the
allocations of variables \emph{p} and \emph{r} are instrumented with
\texttt{\_\_INSTR\_remember} even though pointer analysis evaluated
all dereferences that could point to these variables as safe and no
check was inserted.  We thus do not need to remember the records since
they will not be used anywhere (but they will redundantly prolong the
list of records).

The problem with instrumenting redundant calls to \texttt{\_\_INSTR\_remember*}
functions can be solved by dividing the instrumentation into two phases:
\begin{enumerate}
  \item instrumentation of \texttt{\_\_INSTR\_check\_pointer} calls
  \item instrumentation of \texttt{\_\_INSTR\_remember*} calls
\end{enumerate}

In the first phase, \texttt{\_\_INSTR\_check\_pointer} calls are inserted
according to results of the pointer analysis as described in
Section~\ref{sec:pta}. Additionally, the instrumentation remembers all memory
locations that may have been dereferenced by any instrumented instruction in an
auxiliary list by setting the field \texttt{remember} in the config file.

In the second phase, we insert \texttt{\_\_INSTR\_remember*} calls. For every
memory allocation, we first check whether the underlying memory location has
been remembered in the first phase. \todo{describe checking} If it has, we
create a record for this allocation because a check that can use this record
was inserted in the first stage. Otherwise, this memory allocation is not
instrumented.

The only exception are memory allocations on the heap. As we need to
create these records always due to the check for memory leaks at the
end of \emph{main}, we insert calls to
\texttt{\_\_INSTR\_remember\_malloc} function unconditionally in the
first phase.  Nevertheless, the staged instrumentation can improve
also on memory leaks checking. If some dynamic memory allocation is
instrumented with \texttt{\_\_INSTR\_remember\_malloc} in the first
phase, the \texttt{malloc\_present} flag is set to \emph{true}.  In
the second stage, \texttt{\_\_INSTR\_check\_leaks} is instrumented
only under the condition that the \texttt{malloc\_present} flag was
set to \emph{true}. Therefore we can avoid leaks checking in programs
that do not use dynamic memory allocation.

\todo{example of rules?}

\subsection{Extension With Constant-Time Checks}\label{sec:constant_time}

\lstset{escapeinside={<@}{@>}, columns=fullflexible, basicstyle=\ttfamily, language=llvm, style=nasm}
\begin{figure}[t]
\begin{lstlisting}
%1 = alloca [10 x i32], align 16
%2 = getelementptr inbounds [10 x i32],
       [10 x i32]* %1, i64 0, i64 20
<@{\color[RGB]{0, 135,0} \%3 = bitcast [10 x i32]* \%1 to i8*}@>
<@{\color[RGB]{0, 135,0} call void @\_\_INSTR\_check\_bounds(i8* \%3, i64 40, i2* \%1, i64 4)}@>
store i32 1, i32* %2, align 16
\end{lstlisting}
\caption{Instrumentation of a code with \texttt{\_\_INSTR\_check\_bounds}.}
\label{fig:check_bounds_example}
\end{figure}

The following extension is not directly bound to the stages, but it
can amplify the positive effect of staged instrumentation and
therefore we present it in this section. Let us assume that there is
an instruction that may possibly perform an unsafe dereference
(i.e.~the \texttt{mayBeInvalidDereference} predicate was determined to
hold and thus a check has to be inserted). In some cases, we can
statically find the address and the size of the memory block that is
being dereferenced and therefore we do not need to look for the
corresponding record in the list during run-time. The cases where we
can do this must meet these conditions:
\begin{itemize}
\item the pointer analysis must give us a single memory location where the
dereferenced pointer must point
\item the memory location is not \emph{unknown}, NULL nor invalidated
\item the size of this memory location is constant
\end{itemize}

We query the pointer analysis in a similar way as in
Section~\ref{sec:pta} to get this information.  If the requirements
are not satisfied, the dereference must be instrumented with the usual
\texttt{\_\_INSTR\_check\_pointer}. Otherwise, a call to a
\texttt{\_\_INSTR\_check\_bounds} function is be
inserted. \texttt{\_\_INSTR\_check\_bounds} function takes four
arguments: the address and the size of the dereferenced memory block
(provided by the pointer analysis), the pointer that is being
dereferenced, and the number of dereferenced bytes.  This information
is sufficient to decide whether an out-of-bound error occurs.  The
last requirement demands that the size of the memory location is
constant, thus known at the compile time (and, consequently, known by
the pointer analysis).  This is to avoid the cases where the size
parameter used when allocating the memory may change between the
allocation and the inserted checks.  There are definitely cases where
this optimized instrumentation could be used with variable size
parameter too, but they are a subject of further research.

An example of a constant-time check can be found in
Figure~\ref{fig:check_bounds_example}. On line 6 where the array of ten
integers allocated on line 1 is being dereferenced by \texttt{store}
instruction, the pointer analysis can determine the address and the size of the
array precisely, hence the call to \texttt{\_\_INSTR\_check\_bounds} is
inserted instead of the usual \texttt{\_\_INSTR\_check\_pointer}.

As we already said, this extension is not dependent on the staged
instrumentation. Nevertheless, since the \texttt{\_\_INSTR\_check\_bounds}
function does not use records, no memory locations need to be remembered
because of this check (unlike when using \texttt{\_\_INSTR\_check\_pointer}),
and the second phase of instrumentation may be able to insert fewer calls to
\texttt{\_\_INSTR\_remember} when this extension is used (as can be seen also
in Figure~\ref{fig:check_bounds_example} where no call to
\texttt{\_\_INSTR\_remember} was inserted below the \texttt{alloca}
instruction).

\medskip
Inserting fewer calls to \texttt{\_\_INSTR\_remember} function has
a positive effect on the speed of reachability analysis since the lists
used to keep records are shorter. All the described extensions together
can significantly reduce the number of instrumented instructions
which have also a positive effect on the portion of code possibly
removed by slicing before the reachability analysis.





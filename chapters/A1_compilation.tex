\lstset{
  showstringspaces=false,
  commentstyle=\color{red},
  keywordstyle=\color{blue},
  numbers=none,
  frame=none
}

\section{Archive Structure}

The submitted archive for this thesis contains the repository with the
instrumentation tool that was implemented as part of the thesis.

The directory \emph{llvm-instrumentation} contains sources of the tool in C++
and a python script \emph{instr} for running the instrumentation. There is also
a directory \emph{instrumentation} with configurations for checking memory
safety as described in Chapter~\ref{chap:memsafety} and a directory
\emph{analyses} that contains a plugin performing extended pointer analysis.

The repository can also be found online at \url{https://github.com/staticafi/llvm-instrumentation}.

\section{Compilation and Running of Instrumentation}

To compile and run the instrumentation tool, it is necessary to have
\textsc{CMake} and the LLVM~3.9.1 together with \clang~3.9.1 installed. After
that, go to the \emph{llvm-instrumentation} directory and bootstrap the
libraries for JSON by running the following command:

\begin{lstlisting}[language=bash]
./bootstrap-json
\end{lstlisting}

\noindent Since configuration for checking memory safety is included in the archive, we
also need to bootstrap the library \texttt{dg} for the pointer analysis:

\begin{lstlisting}[language=bash]
./bootstrap-dg
\end{lstlisting}

\noindent Now configure and install the project using:

\begin{lstlisting}[language=bash]
cmake -DCMAKE_INSTALL_PREFIX=. -DCMAKE_INSTALL_LIBDIR=bin .
make install
\end{lstlisting}

\noindent After the successful installation, the tool can be easily used by
running the script \texttt{instr}, that takes a JSON configuration file and the
program in C (or optionally in LLVM bitcode if option \texttt{-{}-bc} is used)
as arguments:

\begin{lstlisting}[language=bash]
./instr options config.json program.bc
\end{lstlisting}

\newpage
\noindent The options can be following:
\begin{itemize}
  \item \texttt{-{}-help} shows the information about running the script
  \item \texttt{-{}-bc} means that the given code is in LLVM
  \item \texttt{-{}-ll} generates the output file also in LLVM that is human-readable
  \item \texttt{-{}-output=FILE} allows to specify the name of the output file
\end{itemize}

\subsection{Compiling and running instrumentation within \symbiotic}

If you want to use the instrumentation for checking memory safety within
\symbiotic, follow the \symbiotic installation
guide\footnote{\url{https://github.com/staticafi/symbiotic/wiki/Symbiotic}}.
The instrumentation is build automatically together with LLVM and other
components of \symbiotic. To enable memory safety instrumentation, run
\symbiotic with the option \texttt{-{}-prp=memsafety}. To see the code after
the instrumentation, run \symbiotic with the option \texttt{-{}-save-files}.
The instrumented code can be found in the file \texttt{*-instr.bc}
in the directory \texttt{symbiotic\_files}.


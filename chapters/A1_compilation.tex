\lstset{
  showstringspaces=false,
  commentstyle=\color{red},
  keywordstyle=\color{blue},
  numbers=none,
  frame=none
}

\section{Archive Structure}

TODO

\section{Compilation and Running of Instrumentation}

To compile and run the instrumentation tool, it is necessary to have
\textsc{CMake} and the LLVM~3.9.1 together with \clang~3.9.1 installed. After
that, bootstrap the libraries for JSON by running the following command in
\texttt{llvm-instrumentation} directory:

\begin{lstlisting}[language=bash]
./bootstrap-json
\end{lstlisting}

\noindent Since configuration for checking memory safety is included in the archive, we
also need to bootstrap the library \texttt{dg} for the pointer analysis:

\begin{lstlisting}[language=bash]
./bootstrap-dg
\end{lstlisting}

\noindent Now configure and install the project using:

\begin{lstlisting}[language=bash]
cmake .
make install
\end{lstlisting}

\noindent After the successful installation, the tool can be used by running the script
\texttt{instr}.

\todo{pokracovani}

\subsection{Compiling and running instrumentation within \symbiotic}

If you want to use the instrumentation for checking memory safety within
\symbiotic, follow the \symbiotic installation
guide\footnote{\url{https://github.com/staticafi/symbiotic/wiki/Symbiotic}}.
The instrumentation is build automatically togehther with LLVM and other
components of \symbiotic. To enable memory safety instrumentation, run
\symbiotic with the option \texttt{-{}-prp=memsafety}. To see the code after
the instrumentation, run \symbiotic with the option \texttt{-{}-save-files}.
The instrumented code can be found in the file with suffix \texttt{-instr.bc}
in the directory \texttt{symbiotic-files}.


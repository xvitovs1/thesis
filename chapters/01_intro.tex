As our lives became highly dependent on computers the question of software
reliability turned to be a serious issue. If we take into consideration the
size and the complexity of today computer programs it is almost impossible to
find bugs in a code manually. The need for automatic program analysis is
therefore increasing.

\textcolor{red}{TODO describe static analysis? what about the other one?
techniques used for static analysis}

One of the techniques used for program analysis is instrumentation.
Instrumenting a program means inserting a new code into an existing code in
order to gather information relevant for program analysis. The new code usually
does not change the behaviour of the program. Instrumentation is for example
used in profilers which insert instructions for gathering information about
time or memory consumption. \textcolor{red}{Another example?}

The aim of this thesis is to present an overview of existing tools for
instrumentation of LLVM IR bitcode and to desing a new tool for LLVM IR
instrumentation. The tool should be configurable and should offer the
possibility to use results of external static analyses to reduce the amount of
newly inserted code. \textcolor{red}{Mention other goals?} 

\textcolor{red}{Chapters descriptions}

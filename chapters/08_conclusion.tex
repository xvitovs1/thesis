Within this thesis, we listed tools that use instrumentation of LLVM~IR and
explained the need for a new tool for instrumentation that is not focused at
only one application, but is general and can be configured by user.

We implemented the tool such that it can be easily configured with two files
provided by user. Moreover, the tool can use results of other programs for
static analyses to get necessary information to reduce the amount of the
inserted code.

We also created a configuration for this tool that can be used to check memory
safety of programs. We improved this configuration with two enhancements:
first, we employed a pointer analysis to find out if it can itself decide
whether a pointer dereference is valid, and then we divided the instrumentation
into stages to leave some of the memory allocations from the instrumentation.
Both of these enhancements led to a smaller amount of inserted code and
therefore to a~faster analysis of the code.

Finally, we evaluated the configuration for checking memory safety by employing
the instrumentation to the tool \symbiotic and running it on a set of
benchmarks from SV-COMP.

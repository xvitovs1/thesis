In this chapter, we evaluate the instrumentation for checking memory safety
that was employed in \symbiotic on benchmarks from SV-COMP~2017.

We used all the benchmarks from the official \emph{MemSafety} category along
with the benchmarks from the subcategory \emph{TerminCrafted}, which was not
included in the official SV-COMP~2017. The benchmarks from these categories are
programs in C and they can contain either no violation of memory safety, or the
following errors:
\begin{itemize}
  \item invalid memory deallocation, e.g. double free error,
  \item invalid pointer dereference,
  \item memory leaks.
\end{itemize}
There were 390 benchmarks in total, \todo{X} of them containing some violation
of memory safety, \todo{Y} of them safe.

\symbiotic first instruments the program according to one of the configurations
described in Chapter~\ref{chap:memsafety}. Since the newly inserted code marks
possible error locations, we reduced the problem to a reachability problem.
\symbiotic slices the program and then runs symbolic executor \klee
that checks the reachability of the error locations. If no error location is
reachable, it answers \emph{true}. If some error location is reachable, it
answeres \emph{false} and gives information about a type of the error (e.g.
invalid dereference or memory leak). If \symbiotic cannot decide the
reachability (a given time limit was exceeded, the tool ran out of memory,
etc.), it answers \emph{unknown}.

All the following measurements were taken on machines with \textit{Intel(R)
Core(TM) i7-3770} CPU that run on 3.40GHz frequency and dispose of 8~GB of
memory. The memory limit was set to 8~GB. We used the utility
\emph{Benchexec}~\cite{Beyer2015} for reliable measurement of consumed
resources.

\section{Comparison of Configurations}
In this section, we compare three configurations introduced in
Chapter~\ref{chap:memsafety}: the basic approach (one-phase instrumentation
without any plugins, denoted as \emph{basic}), the enhancement with a pointer
analysis as a plugin (denoted as \emph{ePTA}) and the staged instrumentation
(denoted as \emph{staged}). We ran \symbiotic with these configurations on the
above mentioned benchmarks from SV-COMP with the CPU time limit set to 120~s.

\begin{table}[t]
\begin{tabular}{l  r   r   r}
 & basic & ePTA & staged \\
 \hline
 size before instrumentation & 155782 & 155782 & 155782 \\
 size after instrumentation  & 322018 & 192468 & 171327 \\
 \hline
 inserted calls (total)    & 166236 & 36686 & 15545 \\
 inserted calls (average)  & 427 & 94 & 39 \\
\end{tabular}
\caption{The comparison of the three configurations for the memory safety
instrumentation. Size is given by the number of instructions of a program.}
\label{tab:numbers}

\end{table}

As for the number of inserted instructions, we present the experimental results
in Table~\ref{tab:numbers}. The first part of the table shows the total size of
the benchmarks expressed with the number of LLVM instructions before and after
instrumentation for each configuration. In the second part, the cumulative and
average number of inserted \texttt{call} instructions is counted. Note that the
total size before instrumentation is 155782 instructions, which is 400
instruction on average. This means that with the basic approach, the final code
after instrumentation is of double size. Evidently, the number of inserted
instructions significantly decreases with each enhancement.

\todo{times}

\begin{table}[h]
\begin{tabular}{l  r   r   r}
 & basic & ePTA & staged \\
 \hline
 true     & 116 & 123  & \textbf{182} \\
 false    & 132 & 133  & \textbf{135} \\
 unknown  & 1   & 1    & 1 \\
 timeout  & 138 & 131  & \textbf{71} \\
\end{tabular}
\caption{The answers given by \symbiotic when using the three configurations for the memory safety
instrumentation.}
\label{tab:answers}

\end{table}

The enhancements had also a positive effect on the answers of \symbiotic as
shown in Table~\ref{tab:answers}. When using the basic approach, \symbiotic did
not manage to decide 138 benchmarks in the given time limit. With a pointer
analysis, it decided 7 more benchmarks. The most significant improvement was
achieved with the staged instrumentation: \symbiotic ran out of time only in 71
cases. The both enhancements had impact especially on the benchmarks that did not
contain any violation of memory safety.

\section{Comparison with Other Tools}

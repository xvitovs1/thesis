In this chapter, we evaluate the instrumentation for checking memory safety
that was employed in \symbiotic on benchmarks from SV-COMP~2017.

We used all the benchmarks from the official \emph{MemSafety} category along
with the benchmarks from the subcategory \emph{TerminCrafted}, which was not
included in the official SV-COMP~2017. The benchmarks from these categories are
programs in C and they can contain either no violation of memory safety, or the
following errors:
\begin{itemize}
  \item invalid memory deallocation, e.g. double free error,
  \item invalid pointer dereference,
  \item memory leaks.
\end{itemize}

\symbiotic first instruments the program according to one of the configurations
described in Chapter~\ref{chap:memsafety}. Since the newly inserted code marks
possible error locations, we reduced the problem to a reachability problem.
\symbiotic optionally slices the program and then runs symbolic executor \klee
that checks the reachability of the error locations. If no error location is
reachable, it answers \emph{true}. If some error location is reachable, it
answeres \emph{false} and gives information about a type of the error (e.g.
invalid dereference or memory leak). If \symbiotic cannot decide the
reachability (a given time limit was exceeded, the tool ran out of memory,
etc.), it answers \emph{unknown}.

All the following measurements were taken on machines with \textit{Intel(R)
Core(TM) i7-3770} CPU that run on 3.40GHz frequency and dispose of 8~GB of
memory. The CPU time limit was set to 900~s and the memory limit was set to
8~GB. We used the utility \emph{Benchexec}~\cite{Beyer2015} for reliable
measurement of consumed resources.

\section{Comparison of configurations}



\section{Comparison with other tools}

In this chapter, we list open-source tools that work with instrumentation of LLVM. In the
first section, we describe a framework for general instrumentation where the
user is somehow able to control the instrumentation in terms of what
instructions will be inserted, etc. In the second section, we list the tools
that use an instrumentation for specified purposes such as checking memory
safety.

\section{LLVMPin Instrumentation Framework}\label{sec:llvmpin}

\llvmpin~\cite{llvmpin} is a framework that simplifies implementing tools for
instrumentation of LLVM~IR. First of all, a user has to implement
an~\textsc{LLVMPin} tool, which is a C++ code that must contain three parts:

\begin{itemize}
    \item Analysis part with definitions of analysis functions and a routine
    for analysis setup. Analysis functions are functions that can be inserted
        into the given code that is supposed to be instrumented to perform an
        analysis of the program.
    \item \texttt{INSTRUMENT} routine that inserts calls to analysis functions
    defined by user.
    \item \texttt{REQUIRE} routine with a list of existing LLVM
    passes\footnote{\url{https://llvm.org/docs/Passes.html}} required
    by the \textsc{LLVMPin} tool.
\end{itemize}

The framework basically consists of two scripts:

\begin{itemize}
    \item \texttt{llvmpin-gen} compiles the \textsc{LLVMPin} tool,
    \item \texttt{llvmpin} runs the instrumentation, generates the new bitcode
    and compiles it to an executable.
\end{itemize}

The framework comes with the API that provides several functions and macros
that are useful for the instrumentation, for example a function
\texttt{INS\_InsertCall} that adds a call to an analysis function. However, a
user still has to implement the eintire \texttt{INSTRUMENT} routine to describe
how to instrument the code, including the procedure that searches for the
instructions that should be instrumented.

To summarize, the \textsc{LLVMPin} framework is one of a few instrumentation
tools for LLVM IR that are general and configurable, but it does not simplify
the work for the user in terms of writing a code that performs the core of the
instrumentation process very much.

\section{Other Tools}

In this section, we describe tools that are not designed for general
instrumentation but use instrumentation over LLVM for various purposes.

\textsc{Clang} compiler provides several sanitizers that use instrumentation to
insert runtime checks for various errors and undefined behaviour. It offers a
memory error detection (out-of-bounds dereferences, use-after-free,
double-free, etc.) with \textsc{AddressSanitizer}~\cite{asan}, that instruments
each memory access with a check to a \textit{shadow memory}. The shadow memory
contains metadata about corresponding program memory, mainly \textit{poisoned
redzones} that mark locations of the memory that should not be accessed.
\textsc{ThreadSanitizer}~\cite{tsan} is used for detection of data races in
parallel programs. The analyzed code is instrumented such that it keeps
information about each memory access and checks whether the access takes part
in a race.  \textsc{UndefinedBehaviourSanitizer} offers undefined behaviour
detection (e.g. signed integer overflows) by instrumenting all operations that
may lead to undefined behaviour with a check. All these sanitizers modify the
code during compilation and insert run-time checks. Disadvantage is a slowdown
and memory overhead these tools introduce, however, the overhead introduced by
other tools that are designed for similar purposes is usually
higher~\cite{tsan,asan}.

\textsc{SAFECode}~\cite{safecode} (Static Analysis For safe Execution of Code)
is a~tool developed at University of Illinois that checks memory safety, such
as accessing invalid memory locations, out-of-bounds accesses, invalid frees
and detection of dangling pointers. Like \clang sanitizers it inserts run-time
checks at compile-time. Moreover, it uses static analysis to minimize the
number of inserted run-time checks.

Another tool that uses instrumentation to check memory safety is
\textsc{Map2Check}~\cite{map2check}. It was inspired by \symbiotic and
\textsc{Leakpoint}~\cite{leakpoint} and it inserts functions that track
allocated memory and assertions that check the safety of operations on
pointers.  Moreover, it can also insert assertions that check integer overflows
after arithmetic operations.  Currently, it does not use any static analysis to
reduce the number of inserted checks. Like \symbiotic, it uses the symbolic
executor \klee~\cite{klee} to check reachability of error locations.

\textsc{Loom}~\cite{loom}  and \textsc{LLVM IR Trace Profiler}~\cite{tracer}
(\textsc{LLVM-Tracer}) are tools used for code profiling. Both of them
dynamically gather information about the program that is being analyzed.

\textsc{Loom} is an LLVM library for instrumentation that enables to insert
lines of code that expose values of LLVM instructions (e.g.~return values of
function calls). The values are then written to a log or printed to the
standard output. In contrast to the above mentioned tools, it is in a sense
configurable. Users can specify what functions should be instrumented, whether
they should be instrumented on the caller site or the callee site, what
structures should be instrumented and whether they should be instrumented on
reads or writes, etc.

\textsc{LLVM-Tracer}~\cite{tracer} is an LLVM pass that, like \textsc{Loom},
instruments the given code to expose values. It traces dynamic register values
as well as memory addresses. The user can again specify what functions should
be instrumented.

All the above described tools can serve well for the purpose they were designed
for. However, if the user wants to change even slightly the semantics of the
instrumented code, the options are very limited. As we consider this to be
a shortcoming, we came up with an idea of a tool for configurable
instrumentation.

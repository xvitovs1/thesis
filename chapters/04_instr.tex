\textcolor{red}{TODO describe how it works + plugins}

The basic idea of our instrumentation tool is simple. Since it is required to
be configurable, it needs to be supplied with two files: a JSON file with
instrumentation rules and a file with definitons of instrumentation functions.
The tool then goes through all instructions of all functions of a given LLVM
code and if it finds a sequence of instructions that match some rule from the
JSON file, it inserts a call to the corresponding function.

\section{Plugins}
The instrumentation can be extended with so-called plugins, which enables us to
use results of some static analyses during the instrumentation. Path to these
plugins can be specified in a JSON configuration file.

\section{Configuration}
The JSON file contains following fields:

\medskip
\begin{itemize}
\item \texttt{file:} Path to a a file with definitions of instrumentation functions
\item \texttt{analyses:} List of paths to analyses plugins
\item \texttt{flags:} List of flags that can be set during instrumentation
\item \texttt{globalVariablesRule:} Rules for instrumenting global
  variables \textcolor{red}{TODO}
\item \texttt{phases:} List of phases. Each phase contains a
  list of \texttt{instructionRules}. Each rule is described with several fields:
  \begin{itemize}
  	\item \texttt{findInstructions}: Sequence of instructions we are searching
	  for. For each instruction in the sequence, we need to fill in an
      \texttt{instruction} field, that specifies a name of the instruction,
      \texttt{returnValue} that enables to remember the return value of the
      instruction in a given variable (can be set to "*" if the return value is not
      needed), and \texttt{operands} that enables either to match the operands or to
      remember the operand values in given variables (see example
      \ref{code:config_example}).
    \item \texttt{newIstruction:} Instruction that is to be inserted
    \item \texttt{in:} Name of a function, in which this rule should be
      applied. Can be set to a value "*" meaning that it should be used in all functions.
    \item \texttt{where:} Specifies the location of insertion. It can be:
      \texttt{before} or \texttt{after} the found sequence of instructions,
      \texttt{entry} (at the entry point of the given function, \texttt{in}
      cannot be set to "*" in this case) or \texttt{return} (before every
      terminator instruction of the given function, \texttt{in} cannot be set to
      "*" in this case).
    \item \texttt{setFlags:} List of pairs \texttt{<flag, value>} that sets
      all \texttt{flags} to a corresponding \texttt{value} if the rule was
      applied. This field is optional.
  \end{itemize}
\end{itemize}
\medskip

\lstset{
    string=[s]{"}{"},
    stringstyle=\color{blue},
    comment=[l]{:},
    commentstyle=\color{black},
}
\begin{lstlisting}
{
  "file": "find_free.c",
  "analyses": ["example_plugin.so"],
  "flags": [""],
  "globalVariablesRule":
  "phases": [
    {
      "instructionRules": [
      ]
    },
    {
      "instructionRules": [
      ]
    }
  ]
}
\end{lstlisting}

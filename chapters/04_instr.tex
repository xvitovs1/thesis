As only one of the tools described in Chapter~\ref{chap:tools} can be used for
an arbitrary instrumentation, we decided to implement a tool for general
instrumentation that would be configurable by a user. Moreover, the user
would not need to implement the core of instrumentation routine as with
\llvmpin (see Section~\ref{sec:llvmpin}).

\smartdiagramset{%
  back arrow disabled=true,
  module minimum width=2cm,
  module minimum height=2cm,
  module x sep=3cm,
  text width=2cm,
  additions={
    additional item offset=0.5cm,
    additional item border color=red,
    additional arrow color=red,
    additional item width=2cm,
    additional item height=2cm,
    additional item text width=3cm
  }
}

\begin{figure}[h]
  \centering
  \begin{tikzpicture}[yscale=0.9, auto,
      block/.style = {
        rectangle, draw=black, thick, text width=8em, text centered,
        rounded corners, minimum height=3em },
      block-sharp/.style = {
          rectangle, draw, thick, text width=8em, text centered, minimum
          height=3em },
      line/.style = { draw, thick, ->, >=stealth },
      line-dashed/.style = { draw, thick, ->, densely dotted, >=stealth }]
    \node [block-sharp] (rules) at (-4.5, -2.75) {\textsf{Instrumentation\\ rules in JSON}};
    \node [block-sharp] (cprogram) at (0, 2.25) {\textsf{Program to be instrumented}};
    \node [block-sharp] (llvmprogram) at (0, 0) {\textsf{Program in LLVM}};
	\node [rectangle split, draw, rectangle split parts=2, text
    width=8em,rounded corners, text centered, fill=blue!30] (instr) at (0, -2.75)
    {\textsf{\textbf{Instrumentation:}} \nodepart{second}
    \parbox{3cm}{\textsf{1. phase \\ 2. phase \\ \vdots}}};
    \node [block-sharp] (instrprogram) at (0, -5.5) {\textsf{Instrumented program in LLVM}};
    \node [block-sharp] (defs) at (4.5, -2.75) {\textsf{Definitions of
    instrumented functions in LLVM}};
    % connect all nodes defined above
    \draw[line] (cprogram) -> (llvmprogram);
    \draw[line-dashed] (llvmprogram) -> (instr);
    \draw[line] (instr) -> (instrprogram);
    \draw[line-dashed] (rules) -> (instr);
    \draw[line-dashed] (defs) -> (instr);
  \end{tikzpicture}
  \caption{The scheme of configurable instrumentation.}
  \label{fig:scheme}
\end{figure}

The basic idea of our instrumentation tool is depicted in
Figure~\ref{fig:scheme}. Since it works on top of LLVM, the given program that
is supposed to be instrumented has to be translated to LLVM first.
Instrumentation is required to be configurable, therefore it needs to be
supplied with two config files created by a user: a JSON file with
instrumentation rules and a~file with definitons of so called
\emph{instrumentation functions} whose calls will be inserted into the code and
a JSON file with instrumentation rules. The instrumentation rules define how
sequences of LLVM instructions should be instrumented with calls of
instrumentation functions.  We only allow to insert call instructions since it
is sufficient as any other demanded instruction can be wrapped in a~called
function.

The instrumentation proceeds in one or more phases, each phase defined by a set
of instrumentation rules.The tool goes through all instructions of the module
loaded from a given LLVM file in each phase and it looks if the instructions
match any instrumentation rule of the current phase. If a match is found, the
rule is applied (i.e.~a~new code is inserted according to the rule). The result
of the instrumentation process is again an LLVM program.

The phased instrumentation is useful because it allows the user to gather some
information in earlier phases and use it in later phases. For example, it is
pointless to insert a check for memory leaks if there is no call to
\emph{malloc} in the program. Therefore, the presence of a call to
\emph{malloc} can be noted in the first phase, and the check for memory leaks
can be inserted in a later phase according to the noted information.

Since this tool for instrumentation was originally implemented in order to
enable checking memory safety in \symbiotic, some of the features described in
this chapter were added just for this purpose. Therefore, the tool might seem
to be concentrated on memory safety. However, it can still be used also for
general instrumentation.

\section{Intuition for Instrumentation Rules}
The instrumentation rules determine what instructions will be instrumented and
when. There is a set of rules for each phase of the instrumentation. The first
part of an instrumentation rule is a sequence of instructions that should be
instrumented. The second part is the definition of the new instruction that
will be inserted if the sequence is matched and the placement of this
instruction (e.g.~after the matched sequence). A user can additionally specify
conditions, under which a~rule should be applied. In order to pass some
information between phases, it is also possible to define flags that can be set
during a rule's application. The detailed description of instrumentation rules
can be found in Section~\ref{sec:config}.

Another useful feature are plugins. Plugins are external analyses that can
evaluate queries on values from the program that is being instrumented. The
queries might be used in conditions for the instrumentation rules.

\section{Conditions}\label{sec:conditions}

To enable conditional instrumentation, the rules can be extended with
conditions. To apply a rule, all conditions for the rule have to be satisfied.

A condition is a query that can take values from the program being verified as
parameters, and a list of expected answers. There are two types of these
conditions:
\begin{itemize}
 \item conditions that check whether a flag is set to a certain value, i.e.~the
 query is the name of the flag and the list of expected results contains the value,
 \item conditions that check answers of plugins to given queries.
\end{itemize}

A condition for a flag is satisfied if and only if the value of the flag equals
one of the values in the list of expected results. A condition for plugins is
satisfied if and only if some answer from a~plugin equals one of the values in
the list of expected answers. That is, all plugins are asked to evaluate the
query and if one of them replies with one of the expected values, the condition
is satisfied.

\section{Flags and Plugins}

As mentioned above, in order to pass some information between phases a user can
define flags in a JSON file with instrumentation rules. When inserting a~call
to a function, it is possible to set a flag and, in a later phase, condition
some rule by a check whether the flag is set to the given value (see
Section~\ref{sec:conditions}).  For example, the user might want to set a flag
\texttt{malloc-instrumented} to \emph{true} when instrumenting a~call to
\emph{malloc} in the first phase of the instrumentation, and to insert a~check
for memory leaks at the end of \emph{main} in the second phase only if the flag
    is set to \emph{true}, i.e.~if there is a call to \emph{malloc} somewhere
    in the program.

The instrumentation module can also be extended by plugins that can reply to
queries, which are basically questions that can be decided by a plugin. Plugins
can reply with a string, i.e.~a~plugin can for example answer
with~\emph{``maybe''}.  To define a list of possible queries, we require the
plugins to implement a given interface. The interface describes a set of
callbacks that correspond to the queries and return string values. For example,
a query ``Can pointers \texttt{p1} and \texttt{p2} point to the same memory
location?'' can correspond to a callback \texttt{pointersCanBeEqual(p1, p2)}.
These functions can be redefined by concrete plugins that are able to decide
the corresponding queries, e.g.~the function \texttt{pointersCanBeEqual} can be
redefined by a plugin performing a pointer analysis, since with this analysis
it is possible to determine whether two pointers can point to the same memory
location. Since each plugin is able to answer to only a subset of the queries
according to its specialization, there are default answers for queries that
cannot be decided by the given plugin. For example, the default answer for the
query ``Can pointers \texttt{p1} and \texttt{p2} point to the same memory
location?'' can be \emph{``unknown''}.

Currently, the following queries with their corresponding meanings are available:
\begin{description}
  \item[isValidPointer(p, offset)] Is a dereference of a pointer \texttt{p}
  valid operation? More precisely, can the memory block to which the pointer
  \texttt{p} points to be accessed at the given \texttt{offset}?
  \item[hasKnownSize(object)] Can we statically determine the size of the memory
         object that is being dereferenced?
  \item[pointsTo(p,a)] Can the given pointer \texttt{p} point to the value \texttt{a}?
  \item[isNull(p)] Can the given pointer \texttt{p} point to NULL?
  \item[isRemembered(value)] Was the given value or a pointer to the value
  stored with a \texttt{remember} field? (see Section~\ref{sec:config} for more
  information)
\end{description}
The default answer for all these queries is \emph{``unknown''} and it is
supposed to be given by the plugins that are not able to evaluate the query. In
Chapter~\ref{chap:memsafety} we implement a plugin performing a pointer
analysis that answers to these queries with answers \emph{``true''},
\emph{``false''} and \emph{``maybe''}.

The list of queries can be easily appended by adding more callbacks that return
string values to the interface and by adding a handler for the added query in the
class \texttt{Analyzer}.


\section{Configuration}\label{sec:config}

In this section, we describe the structure of a JSON file in which
instrumentation rules are defined.

Variables in this configuration file are enclosed by the '<' and '>'
characters. These variables can be used to store some values for later use. For
example, to store an operand of a \texttt{load} instruction to the variable
\texttt{<t>}, a field \texttt{operands} that will correspond to the operands of
\texttt{load} instruction will be set to \texttt{["<t>"]}.  Variable
\texttt{<t>} will contain the value of the operand of \texttt{load} instruction
in the scope of the application of the current rule and it can be used e.g.~as
an argument of the function call that will be inserted or as an argument
of a condition.

A JSON file with instrumentation rules has to contain the following fields:

\begin{center}
\begin{tabular}[h]{>{\bfseries}p{4.2cm} | p{7.8cm}}
  \texttt{file}                & Path to the file with definitions of instrumentation functions. \\
  \hline
  \texttt{analyses}            & List of paths to plugins. \\
  \hline
  \texttt{flags}               & List of flags that can be set during instrumentation. \\
  \hline
  \texttt{phases}              & List of instrumentation phases. Each phase contains a
                                 list of \texttt{instructionRules}. \\
  \hline
  \texttt{globalVariablesRules} & List of rules for instrumenting global
                                 variables.

\end{tabular}
\end{center}

\subsection{\texttt{InstructionRules}}
Each element of \texttt{instructionRules} is JSON object described with
several fields:
\begin{description}
    \item[\texttt{findInstructions}] Sequence of instructions we are searching
    for. For each instruction in the sequence, we need to fill in a field
        \texttt{instruction} that specifies the name of the instruction to be
        matched, \texttt{returnValue} that enables to remember the return value
        of the instruction in a given variable (can be set to "*" if the return
        value is not needed), and \texttt{operands} that enables either to
        match the operands or to remember the operand values in variables. We
        can also optionally fill in fields \texttt{getTypeSize} and
        \texttt{getPointerInfo}. \texttt{getTypeSize} can be used only with
        \texttt{load}, \texttt{store} or \texttt{alloca} instruction and it
        stores the size of the type of the value that is being loaded, stored
        or allocated to the given variable. \texttt{getPointerInfo} can be used
        only with \texttt{load} or \texttt{store} and it stores two values to
        given variables (if possible): size of the allocated memory block to which
        the dereferenced pointer points to and the corresponding \texttt{alloca}
        instruction. Since we can get this information only from a pointer
        analysis, this field can be used only when the analysis is available as
        a~plugin.

        In Figure~\ref{fig:findInstrs_example}, we give an example of a
        \texttt{findInstructions} field. We are searching for a sequence of
        instructions of length one, namely for a \texttt{store} instruction.
        Since \texttt{store} does not return any value, the
        \texttt{returnValue} field is set to "*". The second operand of
        \texttt{store} will be stored in variables \texttt{<t1>} (the first one
        will be skipped as it is set to "*"), the size of the type
        of the value that is being stored will be kept in the variable
        \texttt{<t2>}. \texttt{<t3>} will contain the size of the allocated
        memory to which the pointer in \texttt{<t1>} points to and
        \texttt{<t4>} will contain the \texttt{alloca} instruction that
        allocated this memory.

        \begin{minipage}{\linewidth}
        \lstset{
            basicstyle=\footnotesize,
            string=[s]{"}{"},
            stringstyle=\color{blue},
            comment=[l]{:},
            commentstyle=\color{black},
        }
        \lstinputlisting{examples/findInstrs.json}
        \captionof{figure}{Example of the \texttt{findInstructions} field in a
        JSON configuration file.}
        \label{fig:findInstrs_example}
      \end{minipage}

    \item[\texttt{newInstruction}] A new instruction that is to be inserted. It
    contains two mandatory fields: \texttt{instruction} that specifies a name
    of the new instruction (for now, only \texttt{call} instruction is
    supported), and \texttt{operands} of the instruction. The last operand is
    the name of the function that should be called. For example in
    Figure~\ref{fig:newInstr_example}, the \texttt{newInstruction} field means
    that if the tool decides to apply the rule, a call to a function
    \texttt{\_\_INSTR\_check\_pointer} will be inserted and values of
    \texttt{<t1>} and \texttt{<t2>} will be passed as arguments of the
    function.

     \begin{minipage}{\linewidth}
        \lstset{
            basicstyle=\footnotesize,
            string=[s]{"}{"},
            stringstyle=\color{blue},
            comment=[l]{:},
            commentstyle=\color{black},
        }
        \lstinputlisting{examples/newInstr.json}
        \captionof{figure}{Example of the \texttt{newInstruction} field in a
        JSON configuration file.}
        \label{fig:newInstr_example}
      \end{minipage}

    \item[\texttt{in}] Name of a function, in which this rule should be
    considered for application. For example, \texttt{\textcolor{blue}{"in"}:
    main} means that this rule will be checked only in the \emph{main}
    function. It can also be set to a value "*" meaning that it should be used
    in all functions.

    \item[\texttt{where}] Specifies the location of insertion. It can be:
    \emph{``before''} or \emph{``after''} the found sequence of instructions,
    \emph{``entry''} (at the entry point of the given function) or \emph{``return''}
    (before every \texttt{return} instruction of the given function).

    \item[\texttt{conditions}] List of conditions that have to be satisfied
    (see Section~\ref{sec:conditions}). A condition consists of fields
    \texttt{query} and \texttt{expectedResult}. The \texttt{query} is a list
    where the first element is the name of a query and other elements are
    parameters passed to the query, the \texttt{expectedResult} is a list of
    expected results of the query.

    In Figure~\ref{fig:conditions_example}, a \texttt{conditions} list contains
    two conditions that have to be satisfied for the rule to be applied. The
    first condition is a query for plugins. This condition will be satisfied if
    at least one of the plugins will answer \emph{``false''} or \emph{``maybe''} to
    the query \texttt{isValidPointer} with values of \texttt{<t1>} and
    \texttt{t2} passed as parameters. The second condition will be satisfied if
    the flag \texttt{testFlag} is set to \emph{``true''}.

    \begin{minipage}{\linewidth}
        \lstset{
            basicstyle=\footnotesize,
            string=[s]{"}{"},
            stringstyle=\color{blue},
            comment=[l]{:},
            commentstyle=\color{black},
        }
        \lstinputlisting{examples/condition.json}
        \captionof{figure}{Example of the \texttt{conditions} field in a
        JSON configuration file.}
        \label{fig:conditions_example}
      \end{minipage}

    \item[\texttt{setFlags}] List of pairs \texttt{[flag, value]} that sets all
    \texttt{flags} to the corresponding \texttt{value} if the rule was applied.
    This field is optional. We give an example of this field in
    Figure~\ref{fig:setFlags_example}, which sets the flag \texttt{loadFlag}
    to \emph{``true''} and the flag \texttt{testFlag} to \emph{``false''}.

     \begin{minipage}{\linewidth}
        \lstset{
            basicstyle=\footnotesize,
            string=[s]{"}{"},
            stringstyle=\color{blue},
            comment=[l]{:},
            commentstyle=\color{black}
        }
        \lstinputlisting{examples/set_flags.json}
        \captionof{figure}{Example of the \texttt{setFlags} field in a
        JSON configuration file.}
        \label{fig:setFlags_example}
      \end{minipage}

    \item[\texttt{remember}] The name of a variable, whose value should be stored in an
    auxiliary list, e.g. \texttt{\textcolor{blue}{"remember"}:"<t1>"}. If some
    rule is conditioned by a~query \texttt{isRemembered} with the variable or a
    pointer to the variable as its parameter, the condition will be satisfied.
\end{description}

\subsection{\texttt{globalVariablesRules}}

\texttt{globalVariablesRules} can be used for instrumenting global
variables. The rules are applied on all declarations of global variable. Each
element of \texttt{globalVariablesRules} is a JSON object with the following
fields:
\begin{description}
    \item[\texttt{findGlobals}:] Contains a mandatory field
    \texttt{globalVariable} that stores the value of a global variable to the
    given variable, and an optional field \texttt{getTypeSize} that gets the size
    of the type of the global variable. The field \texttt{findGlobals} in
    Figure~\ref{fig:findGlobals_example} stores the value of a global variable
    to the variable \texttt{<t1>} and stores the size of its type in
    \texttt{<t2>}.

    \begin{minipage}{\linewidth}
        \lstset{
            basicstyle=\footnotesize,
            string=[s]{"}{"},
            stringstyle=\color{blue},
            comment=[l]{:},
            commentstyle=\color{black},
        }
        \lstinputlisting{examples/findGlobals.json}
        \captionof{figure}{Example of the \texttt{findGlobals} field in a
        JSON configuration file.}
        \label{fig:findGlobals_example}
      \end{minipage}

    \item[\texttt{newInstruction}:] The same as in \texttt{instructionRule}.
    \item[\texttt{in}:] Name of a function, at the beginning of which the new
    instruction will be inserted.
    \item[\texttt{conditions}:] The same as in \texttt{instructionRule}.
\end{description}

The functions whose calls are instrumented into a code must be defined in a
file specified by the \texttt{file} field in the JSON configuration. The
function can be implemented in any language that can be translated into LLVM,
since the file with the functions is compiled to LLVM first and after a
successful instrumentation, it is linked to the instrumented module.

\section{Applying Rules}

In this section, we describe the process of applying the instrumentation rules
on a module of a given LLVM program as depicted in
Algorithm~\ref{alg1}.


First, the rules from the JSON file are parsed. If there are some paths to
plugins specified, the plugins are loaded and the instrumentation part begins.

\begin{figure}[t]
\newcounter{nalg}[chapter] % defines algorithm counter for chapter-level

\begin{algorithm}[caption={Applying the instrumentation rules.}, label={alg1},columns=fullflexible]
 input: module, config.json
 output: instrumented llvm module
 begin
   config $\gets$ parse config.json
   foreach phase in config
   begin
      foreach function in module
      begin
         foreach rule in phase where $\texttt{in = "*"}$ or $\texttt{in =}$ name of the function
         begin
            if rule matches the sequence of instructions of the current function
               and conditions are satisfied then insert the new instruction
         end
      end
   end
   apply rules for global variables
   return instrumented module
 end
\end{algorithm}
\end{figure}

The instrumentation is implemented to be staged, therefore the first step is to
process all the phases one after each other in the following way: iterate over
functions from the loaded module and check rules from the phase being processed
that are relevant for the current function. That is, take into consideration
only rules where \texttt{in} field is set to "*" or to the name of the
function.  As for the other rules, compare the
sequences of instructions defined by these rules (field
\texttt{findInstructions}) with instructions of the current function and if
some rule matches the sequence, begin to process the rule.

The sequences are compared with a simple algorithm: iterate over the
instructions of the current function and if the first instruction from the
sequence in \texttt{findInstructions} does not match the current instruction
from the program, skip this rule. It they match, try to match the second
instruction from the sequence with the following instruction from the program.
If they do not match, skip this rule. If they match, try to match the third
instruction from the sequence with the following instruction from the program,
and so on. There is currently no need for a more efficient algorithm, since the
sequences we are looking for are usually very short, mostly of length one.

To process the rule, first of all, the values of operands and return values of
the matched instructions that are noted by the rule to be saved are stored to
given variables. If the fields \texttt{getTypeSize} or \texttt{getPointerInfo}
are present, the demanded information is stored to the given variables. Then
all conditions are checked as described in Section~\ref{sec:conditions}. If
they are satisfied, the rule is ready to be applied. The new \texttt{call}
instruction that is to be inserted is created according to
\texttt{newInstruction} field and injected before or after the matched sequence
of instructions, or at the beginning or at the end of a given function
according to the \texttt{where} field. For now, we support only inserting
\texttt{call} instructions, however, it does not limit the user in any way.
Since functions whose calls will be inserted are defined by the user, all other
instructions can be inserted through these functions.

\lstset{escapeinside={<@}{@>}, columns=fullflexible, basicstyle=\ttfamily, language=llvm, style=nasm}
\begin{figure}[t]
\begin{lstlisting}
%result = add i32 5, 10
<@{\color[RGB]{0, 135, 0} \%1 = bitcast i32 \%result to i64}@>
<@{\color[RGB]{0, 135, 0} call void @\_\_INSTR\_log\_addition(i64 \%1)}@>
\end{lstlisting}
\caption{Instrumentation of a code where a bitcast is needed. The
\texttt{\_\_INSTR\_log\_addition} function takes a 64-bit integer as an
argument. The 32-bit integer \texttt{result} thus needs to be converted before
it is passed as the argument.}
\label{fig:bitcast_example}
\end{figure}
Since the parameters of the called function might be of different types than
values that are passed as these parameters, we might need to convert them
before the insertion of the new call. In this case, we first insert the
\texttt{bitcast} instruction that performs the needed conversion. For example,
in Figure~\ref{fig:bitcast_example}, the function
\texttt{\_\_INSTR\_log\_addition} that is to be called takes a 64-bit integer
as a parameter, but the value that we want to pass as the parameter
(\texttt{\%result}) is a 32-bit integer. Therefore, the instruction \texttt{bitcast
i32 \%result to i64} will be inserted before the new call instruction.

If the rule is successfully applied, flags are set according to the field
\texttt{setFlags} (if present). If the field \texttt{remember} is defined, the
given value is stored to an auxiliary list and can be used in later phases of
instrumentation.

At the very end, after each phase from the configuration was processed, rules
for global variables are applied. That is, for each rule for global variables,
    a new instruction is inserted at the beginning of the function given in the
    \texttt{in} field if conditions are satisfied.

\lstset{
    basicstyle=\footnotesize,
    string=[s]{"}{"},
    stringstyle=\color{blue},
    comment=[l]{:},
    commentstyle=\color{black},
    escapeinside={<@}{@>}
}

\begin{figure}[h]
\lstinputlisting{examples/json_example.json}
\caption{Example of an \texttt{instructionRule} in a JSON configuration file.}
\label{fig:json_example}
\end{figure}

In Figure~\ref{fig:json_example}, we can see an example of the
\texttt{instructionRule}. In this case, every time the tool comes across a
\texttt{load} instruction in any function (\texttt{in} is set to "*"), the only
operand of \texttt{load} is stored to a variable \texttt{<t1>}, size of the
type that is being loaded is stored to a variable \texttt{<t2>} and the condition
\texttt{isValidPointer} is checked by all plugins (i.e.~a~function
\texttt{isValidPointer} is called with arguments \texttt{<t1>} and
\texttt{<t2>} by all plugins). If some plugin answered either \emph{``false''} or
\emph{``maybe''}, the condition is satisfied and a call of the function
\texttt{\_\_INSTR\_check\_pointer} with arguments \texttt{<t1>} and
\texttt{<t2>} is inserted \emph{before} the current \texttt{load} instruction.
After a successful application of this rule, flag \texttt{loadPresent} is set
to \emph{true}.

\begin{figure}[h]
\lstinputlisting{examples/json_example2.json}
\caption{Example of a \texttt{globalVariablesRule} in a JSON configuration file.}
\label{fig:json_example2}
\end{figure}

An example of a \texttt{globalVariablesRule} can be found in
Figure~\ref{fig:json_example2}. The value of a global variable is stored to a
variable \texttt{<t1>} and the size of the type of this global variable is stored to
a variable \texttt{<t2>}. Plugins are asked whether the given condition
\texttt{isRemembered} with the argument \texttt{<t1>} holds and if the answer is
positive, a new call of the function \texttt{\_\_INSTR\_remember\_global} with
arguments \texttt{<t1>} and \texttt{<t2>} is inserted at the beginning of
\emph{main}.

\section{Unreachable Functions}

Sometimes the program might contain definitions of functions that are never
called. Such functions can be omitted from the instrumentation process because
the code inserted to these functions would never be executed. To determine what
functions can potentially be called, we first need to build a call graph from
the given program. Then we search the call graph with the breadth-first search
algorithm starting from the node that represents the \emph{main} function. All
functions (or nodes precisely) that were visited during the search are marked
as reachable. All the other functions are never called and are not
instrumented. A~pointer analysis is necessary for building this graph because
functions can be called indirectly through function pointers. Therefore, this
functionality is available only if there is a pointer analyses plugin.

\section{Logging}

To sum up the instrumentation process for a concrete file, each operation is
logged with a logger. The logger notes following operations:
\begin{itemize}
  \item parsing configuration
  \item loading plugins (also lists the names of the loaded plugins)
  \item the beginning and the end of each phase
  \item omiting a function from instrumentation
  \item rule application and instruction insertion
  \item the end of instrumentation (also logs the name of the resulting file)
\end{itemize}
The logger also logs errors such as an error when parsing configuration,
invalid path to a plugin, a non-existing function instrumentation, etc. Logs
can be found in \emph{log.txt} file that is generated at the end of the
instrumentation process.

